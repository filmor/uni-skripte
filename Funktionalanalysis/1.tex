\section{Normierte Räume}
Sei $\menge{K}=\menge{C}$ oder $\R$
\begin{dfn}
	Sei $X$ ein $\menge{R}$-Vektorraum.
	Eine Abbildung $ p:X\rightarrow \menge{R} _0^+ $ heißt Halbnorm, falls
	\begin{itemize}
		\item $p(\lambda x)=|\lambda |p(x),\quad \forall \lambda \in \menge{K}, x \in X $
		\item $p(x+y) \le p(x)+p(y),\quad \forall x,y \in X$
	\end{itemize}
	heißt Norm, falls
	\begin{itemize}
		\item $p(x)=0 \Rightarrow x=0$
	\end{itemize}
	$(X,p)$wird (Halb-)Normierter Raum genannt.
\end{dfn}
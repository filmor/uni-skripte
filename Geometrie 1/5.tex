Sei $f : M \to \R^n$ eine Immersion. $g$ ist die \emph{erste Fundamentalform} zu
$F$\dots

$v,w \in T_p M : \sprod[p]{v,w} := \sprod{\d f(p)v, \d f(p)w}$. Falls
$(x, U)$ eine Karte mit zugehörigen Vektorfeldern
\begin{align}
  X_1 = \frac{\partial}{\partial x^1}, \dots, X_m = \frac{\partial}{\partial x^m}
\end{align}
Dann ist $g_{ij}(p) := \langle X_i(p), X_j(p)\rangle_p, p\in U$ die
\emph{Fundamentalmatrix} von $g$ bezüglich $X_1, \dots, X_m$.

Falls $X = \xi^iX_i, Y = \eta^iX_i$ Vektorfelder von $M$ über U sind, dann ist
$\langle X, Y\rangle = \xi^i\eta^jg_{ij} : U \to \R$.

Falls $c : I \to U \subset M$ glatt, so ist
\begin{align*}
  L(f\circ c) &= \int_I\norm{(f\circ c)'(t)}\d t \\
              &= \int_I\norm{\d f(c(t)) \cdot c'(t)}\d t \\
              &= \int_I\norm[c(t)]{c'(t)} \d t \\
              &= \int_I\norm[c(t)]{(c^i)'(t)X_i(c(t))} \d t \\
              &= \int_I\left( (c^i)'(t)(c^j)'(t)g_{ij}(c(t))
              \right)^{\frac{1}{2}}\d t \\
              &=: L_g(c)
\end{align*}
die Länge von $c$ bezüglich $g$.

Zu $p,q\in M$ setzen wir
\begin{align}
  d(p,q) &= d_g(p,q) &= \inf\mdef[L(c)]{c:[a,b]\to M \text{ ist stückweise glatt
  mit } c(a) = p \text{ und  } c(b) = q}
\end{align}
Wir definieren also den Abstand als das Infimum der Länge aller glatten Wege von
$p$ nach $q$ in der Mannigfaltigkeit.

\begin{stz}
  $d$ ist eine Metrik auf $M$, die die auf $M$ vorgegebene Topologie induziert.
  \begin{proof}
    Symmetrie und $\triangle$-Ungleichung sind klar. Für alle $p,q \in M$ ist
    $d(p,q)\geq 0$ (per def.).

    Seien $p,q\in M, p\neq q$. Wähle eine Karte $(x, U)$ von $p$ mit $q\notin U$
    und $x(p) = 0$. Nun sind die $g_{ij} : U \to \R$ (wie oben) glatt und
    $\left( g_{ij}(p) \right)$ ist positiv definit $\forall p\in U$.

    Wähle $\delta > 0$ so, dass $\mdef[u\in \R^n]{\norm{u} \leq \delta} = \bar
    B_\delta (0) \subset \R^m$ enthalten ist in $x(U) =: U'$. Zu diesem $\delta$
    gibt es dann ein $\epsilon > 0$ mit
    \begin{align}
      \epsilon^2 \delta_{ij}v^iv^j &\leq g_{ij}(p')v^iv^j \\
      &\leq \frac{1}{\epsilon^2} \delta_{ij}v^iv^j
      \label{eqn:5.g_ungleichung}
    \end{align}
    für alle $v = (v^1,\dots,v^m)\in \R^m$ und alle $p'\in U$ mit $x(p')\in \bar
    B_\delta (0)$.

    Falls nun $c : [a, b] \to M$ eine stückweise glatte Kurve von $p$ nach $q$
    ist, $c(a) = p$ und $c(b) = q$, dann gibt es ein kleinstes $t_0 > a$ mit
    $\norm{x\circ c)(t)}$, $0 \leq t < t_0$ und $\norm{(x\circ c)(t_0)} =
    \delta$. Daher ist
    \begin{align}
      L_g(c) &\ge L_g(c\vert_{[a,b]}) = \int_0^{t_0} \sqrt{g_{ij}(c(t))(x^i\circ
      c)'(t)(x^j\circ c)'(t)}\d t \\
             &\ge \epsilon \int_0^{t_0}
             \sqrt{\sum_{i=1}{n}(x^i\circ c)'(t)(x^i\circ c)'(t)}\d t \\ &= \epsilon
             L\left( (x\circ c)\vert_{[a,t_0]} \right) \ge \epsilon\delta > 0
             \label{eqn:5.lg_ungleichung}
    \end{align}
    Dies folgt aus Ungleichung \ref{eqn:5.g_ungleichung} und daraus, dass
    $(x\circ c)\vert_{[a,t_0]}$ ein Weg ist, der $0$ mit einem Punkt auf der
    $\delta$-Sphäre verbindet.

    Damit ist $d_g(p,q) \ge \epsilon\delta$, also ist $d_g$ eine Metrik.


    Sei nun $V \subset M$ offen und $p\in V$. Wähle Karte $(x, U)$ um $p$ mit
    $x(p)$ und $U \subset V$. Wähle $\epsilon$ und $\delta$ wie oben. Falls dann
    $q\in M$ mit $q\notin U$ (oder $q\in U$ mit $\norm{x(q)}\ge 1$), so ist
    $d(p,q) \ge \epsilon\delta$ mit dem selben Argument wie oben. Daher ist
    $\mdef[q\in M]{d_g(p,q)<\epsilon\delta}\subset U \subset V$.

    Sei nun umgekehrt $p\in M$ und
    \begin{align}
      B = \mdef[q\in M]{d_g(p,q) < \alpha}
      \label{}
    \end{align}
    für ein $\alpha > 0$. Dann enthält $B$ eine Umgebung (bezüglich der
    vorgegebenen Topologie) von $p$. Wähle dazu $(x, U)$, $\epsilon$ und
    $\delta$ wie oben. Sei $u = x(q)$ mit $\norm u \ge \delta$ und sei $c(t) =
    x^{-1}(tu)$, $0 \le t \le 1$. Nach Ungleichung \ref{eqn:5.lg_ungleichung}
    ist $L_g(c) = \int_0^1 \sqrt{g_{ij}(c(t))u^iu^j}\d t \le
    \frac{\norm u}{\epsilon}$. Daher ist $\mdef[q\in U]{\norm{x(q)} <
    \epsilon\alpha} \subset B$.
  \end{proof}

  \begin{bsps}
    \begin{enumerate}
      \item $\id : \R^m \to \R^m$
      \item $S^m = \mdef[x\in \R^{m+1}]{\norm x = 1}$, $f : S^m\to\R^{m+1}$ die
        Inklusion.
        
        \emph{Behauptung:} $d(p,q) = \sphericalangle(p,q)$

        Sei dazu $p\in S^m$. Parametrisiere Punkte in $S^m$ durch $q = (\cos t)p
        + (\sin t)y, y\in p^\perp, \norm y = 1$. Für $0< t < \pi$ ist dann $q\in
        S^m \setminus \mdef{\pm p}$. Auf dieser Teilmenge erklären wir
        Vektorfelder $V$ durch $V(q) = -(\sin t)p + (\cos t)y$. Es ist $\norm
        V(q) = 1$. Sei nun $c:[a,b] \to S^m$ eine stückweise glatte Kurve von
        $p$ nach $q$.
        oBdA ist $c([a,b])\subset S^m\setminus\mdef{\pm p}$ und 
    \end{enumerate}
  \end{bsps}:w
\end{stz}

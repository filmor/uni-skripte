\renewcommand{\d}{\ensuremath{\mathrm d}}

\section{Untermannigfaltigkeiten in euklidischen Räumen}

\begin{dfn}
Sei $f : M \to \R^n$ (wobei $M$ eine Mannigfaltigkeit ist) eine
\emph{Immersion}, das heißt $df(p) : T_p M \to \R^n$ (das Differential als
Abbildung aus dem Tangentialraum nach $\R^n$) ist injektiv für alle $p\in M$.
Ist $n=1$ also $f$ eine Kurve, dann ist dieser Begriff gleichbedeutend damit,
dass die Kurve \emph{regulär} ist.

\begin{bsp}
  \begin{enumerate}
    \item[Graphen] Sei $g: U \to \R^k$ glatt mit $f:U\to\R^n$ und $n = m + k$,
      $U\subset \R^m$ offen. $g$ ist definiert durch $f(p) = (p,g(p))$
      % Bild: U in der xy-Ebene, f(U) darüber, g(p) Höhe

      Dann gilt $\d f(p)(v) = (v, \d g(p)(v))$, also ist $\d f(p)$ injektiv.
    \item[Drehflächen] (schon diskutiert)
    \item Sei $M\subset\R^n$ Untermannigfaltigkeit. Dann ist die Inklusion
      $\iota : M \to \R^n$ eine Immersion.
  \end{enumerate}
\end{bsp}
\end{dfn}

Sei $I = [a,b]$ und $c : I \to M$ glatt. Dann ist $f\circ c : I \to \R^n$ glatt
der Länge
\begin{align}
  L(f\circ c) &= \int_a^b\left|(f\circ c)'(t)\right| \d t \\
              &= \int_a^b\left|\d f(c(t))(c'(t))\right| \d t
\end{align}


Wir möchten nun diese Kurvenlänge schon in $M$ bestimmen können. Dazu erklären
wir auf $T_pM$ für alle $p\in M$ folgendes Skalarprodukt:
\begin{align}
  g_p(v, w) := \langle v,w\rangle_p = \langle \d f(p)\cdot v, \d
  f(p) \cdot w\rangle \quad v,w \in T_pM
\end{align}
Die zugehörige Norm bezeichnen wir mit $\norm[p]{\cdot}$, also
$\norm[p]{v} = \norm{\d f(p)\cdot v}, v\in T_p M$. Damit ist die
Länge der Kurve:
\begin{align}
  L(f\circ c) = \int_a^b\left|\d f(c(t))\cdot c'(t)\right|\d t =
  \int_a^b\left|c'(t)\right|_{c(t)}\d t
\end{align}

Die innere Geometrie von $M$ wird also durch die Familie $g_p$ von
Skalarprodukten bestimmt.


Sei nun $U\subset M$ offen und $x : U \to U' \subset \R^m$ eine Karte für $M$
über $U$. Zugeordnet zu dieser Karte haben wir die Koordinatenvektorfelder
\begin{align}
  X_j = \frac{\partial}{\partial x^j}, \quad ^\leq j \leq m
\end{align}

Dann ist $(X_1(p), \dots, X_m(p))$ für alle $p\in M$ eine Basis von $T_p M$.

Wir definieren Funktionen $g_{ij} : U\to \R$ durch
\begin{align}
  g_{ij}(p) &:= \langle X_i(p), X_j(p)\rangle_p \\
            &= \langle \d f X_i(p), \d f X_j(p)\rangle \\
            &= \langle \frac{\partial}{\partial x^i(p)},
            \frac{\partial}{\partial x^j(p)} \\
            &= \langle f_i(p), f_j(p) \rangle
\end{align}
Damit sehen wir, dass die $g_{ij}$ glatt sind.

Seien $X,Y$ Vektorfelder auf $U$,
\begin{align}
  X(p) = \xi^i(p)X_i(p), \qquad Y(p) = \eta^i(p)X_i(p)
\end{align} (Einsteinsche Summenkonvention!)

Nach den Definitionen sind $X$ und $Y$ glatt genau dann, wenn ihre
\emph{Hauptteile} $\xi, \eta$ bezüglich x glatt sind.

Wir ethalten
\begin{align}
  (X(p), Y(p))_p &= (\xi^i(p)X_i(p), \eta^j(p)X_j(p))_p
  &= \xi^i\eta^jg_{ij}
\end{align}

Wir haben also das Skalarprodukt bezüglich des Differentials zurückgezogen.

Wir nennen die Familie $g = (g(p))_{p\in M}$ von Skalarprodukten die \emph{erste
Fundamentalform} von $f$.


\begin{dfn}[Lie-Gruppe]
  Eine \emph{Lie-Gruppe} ist eine Gruppe mit einer differenzierbaren Struktur,
  \begin{align}
    g\mapsto g^{-1}, \quad (h,g)\mapsto h\cdot g
  \end{align}
  $C^\infty$-Abbildungen sind.

  \begin{bsps}
    \begin{itemize}
      \item $\mdef[A\in \R^{n\times m}]{A^tBA=B}$
      \item $O, Gl, SO, Sl, U$
    \end{itemize}
  \end{bsps}
\end{dfn}


\begin{dfn}
   Eine differenzierbare Abbildung $f : M \to N$, $M,N$ differenzierbaren
   Mannigfaltigkeiten, heißt \emph{Submersion} falls $\d f : T_x M \to
   T_{f(x)}N$ surjektiv ist.
\end{dfn}

\begin{dfn}
  Sei $G$ eine Gruppe. Eine (linke) \emph{Aktion} ist eine Abbildung $\phi:
  G\times M \to M$ ($M$ Mannigfaltigkeit) sodass
  \begin{itemize}
    \item $\varphi(g_1, \varphi(g_2, m)) = \phi(g_1g_2, m)$
    \item $\varphi(e, m) = m$
  \end{itemize}

  $\varphi$ heißt \emph{frei}, falls aus $gm = m$ immer $g = e$ für alle $m\in
  M$ folgt.

  $\varphi$ heißt \emph{eigentlich}, falls für alle kompakten Mengen
  $K\subset M$ die Menge $\mdef[g\in G]{gK\cap K \neq \emptyset}$ einen
  kompakten Abschluss hat.

  $\varphi$ heißt \emph{transitiv}, falls für alle $m_1, m_2 \in M$ ein $s\in G$
  mit $gm_1 = m_2$ existiert. Der Quotient $M/G = \mdef[\mdef[gm]{g\in G}]{m\in
  M}$ ist die Menge der Orbits.
\end{dfn}

\begin{stz}
  $M$ sei eine Mannigfaltigkeit, $G$ eine Lie-Gruppe die glatt auf $M$ wirkt. Es
  gibt eine differenzierbare Struktur auf $M/G$ genau dann, wenn die Aktion frei
  und eigentlich ist.
  \begin{proof}
    Gibt's hier nicht :/
  \end{proof}
\end{stz}


\begin{stz}
  Die folgenden Aussagen sind äquivalent:
  \begin{itemize}
    \item Die Lie-Gruppe $G$ wirkt transitiv und glatt auf $M$, mit $H =
      \mdef[h\in H]{hm = m}$
    \item $M$ ist diffeomorph zu $G/H$
  \end{itemize}
\end{stz}

\begin{bsp}
  Sei $p\leq n$, $S_{n,p} = \mdef[(x^k)_{k=1\dots p}]{}$
  
\end{bsp}

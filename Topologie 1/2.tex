\section{Topologie}

Notation: X eine Menge, $\mathcal{P}(X)$ die Potenzmenge von X.

\begin{dfn}
    Eine \emph{Topologie} $\tau$ auf einer Menge X ist eine Teilmenge
    $\tau\subset\mathcal{P}(X)$ mit
    \begin{enumerate}
        \item $\emptyset, X \in \tau$
        \item Ist $I\subset \tau$, dann $\bigcup_{U\in I} U \in\tau$
        \item $\mdef{U_1, \ldots, U_n} \subset \tau$, dann
            $\bigcup_{i=1}^n U_i \in \tau$
    \end{enumerate}
    Das Paar $(X, \tau)$ ist dann ein \emph{topologischer Raum}, $U\in\tau$ heißt
    eine \emph{offene Teilmenge von $X$}.
\end{dfn}

\begin{dfn}
    Sei $X$ eine Menge, $\tau_1$ und $\tau_2$ Topologien auf $X$.
    \begin{itemize}
        \item $\tau_1$ heißt \emph{feiner} als $\tau_2$, falls $\tau_2\subset\tau_1$
        \item $\tau_1$ heißt \emph{strikt feiner} als $\tau_2$, falls
            $\tau_2\subsetneq\tau_1$
        \item $\tau_1$ und $\tau_2$ heißen \emph{vergleichbar}, falls
            $\tau_1\subset\tau_2$ oder $\tau_2\subset\tau_1$
    \end{itemize}
\end{dfn}

\newtheorem{bsps}[dfn]{Beispiele}
\begin{bsps}
    \begin{enumerate}
        \item $(X,d)$ metrischer Raum und sei $\tau_d$ die Menge der offenen
            Teilmengen von $X$. Dann ist $\tau_d$ die \emph{metrische Topologie}
            auf $(X,d)$.
        \item $\mathcal{P}(X)$ ist eine Topologie auf $X$ (diskrete Topologie). Es
            ist die feinste aller Topologien.
        \item $\tau = \mdef{\emptyset, X}$ definiert eine Topologie auf X. Sie
            heißt die \emph{grobe Topologie}.
        \item $X=\mdef{1,2,3}$. Dann sind $\tau_1=\mdef{\emptyset,X,\mdef{1}}$ und
            $\tau_2=\mdef{\emptyset, X, \mdef{2}, \mdef{2,3}}$ Topologien auf $X$,
            die nicht vergleichbar sind. Dagegen ist
            $\mdef{\emptyset, X, \mdef{1}, \mdef{2}, \mdef{2,3}}$ keine Topologie.
        \item Sei $X$ eine Menge und
            $\tau = \mdef{\emptyset}\cup\mdef[U\subset X]{X\setminus U \ \text{ist endlich}}$.
            Das definiert eine Topologie auf $X$, die \emph{Endliche-Komplemente-Topologie}
    \end{enumerate}
\end{bsps}

Im Fall von metrischen Räumen hat man die Topologie mit Hilfe der offenen Kugeln
definiert, ähnlich\begin{dfn}
    Sei $X$ eine Menge. Eine \emph{Basis} $B$ einer Topologie auf $X$ ist eine
    Teilmenge $B\subset\potmenge{X}$ mit
    \begin{itemize}
        \item $\forall x\in X\ \exists U\in B$ mit $x\in U$
        \item Falls $U_1,U_2\in B$ und $x\in U_1 \cap U_2$, dann existiert ein
            $U_3 \in B$ mit $x\in U_3 \subset U_1\cap U_2$
    \end{itemize}
\end{dfn}

\begin{stz}
    Sei $X$ eine Menge, $B\subset\potmenge{X}$ eine Basis. Sei
    $\tau\subset\potmenge{X}$ mit $U\in\tau$ genau dann, wenn es für jedes $x\in U$
    ein $V\in B$ gibt, mit $x\in V\subset U$. Dann ist $\tau$ eine Topologie auf $X$.
    Sie heißt die \emph{von $B$ erzeugte Topologie}.

    \begin{proof}[Beweis]
        \begin{itemize}
            \item $\emptyset\in\tau$ klar.
            \item $X\in\tau$ klar (2.4).
            \item bla
        \end{itemize}
    \end{proof}

\end{stz}

\begin{lem}
    Sei $X$ eine Menge, $B\subset\potmenge{X}$ eine Basis und $\tau$ die von $B$
    erzeugte Topologie. Dann gilt: \[\tau = \mdef[\cup_{U\in I}U]{I\subset B}\]

    \begin{proof}[Beweis]
        Analog zu Lemma 1.35 (ref).
    \end{proof}
\end{lem}

\begin{lem}
    Sei $(X, \tau)$ ein topologischer Raum. Sei $B\subset \tau$ mit der Eigenschaft:
    F"ur jedes $U\in \tau$ und jedes $x\in U$ existiert $V\in B$ mit $x\in V\subset U$.
    Dann ist $B$ eine Basis, die $\tau$ erzeugt.

    \begin{proof}
        \begin{enumerate}
            \item $X$ ist offen, also $\forall x\in X \exists V\in B$ mit $x\in V$.
                Damit gilt 2.4(1) (ref).
            \item Sei $x\in V_1\cap V_2$ mit $V_1, V_2 \in B \Rightarrow V_1, V_2 \in \tau$.
                also $x\in V_1\cap V_2$ offen, also $\exists V_3$ mit
                $x\in V_3\subset V_1\cap V_2$ (2.4(2) ref).
            \item Sei $\tau'$ die Topologie, die von $B$ erzeugt wird. Wir m"ochten zeigen:
                $\tau' = \tau$.\\
                Sei $U\in\tau'$, dann ist $U$ Vereinigung von Elementen in $B$
                (Lemma 2.6 ref),
                aber alle solche sind offen in $\tau$, also gilt $U\in\tau$.\\
                Umgekehrt: Sei $U\in\tau$, $x\in U$, dann existiert $V\in B$ mit
                $x\in V\subset U$, also ist $U$ offen bez"uglich $\tau'$.
        \end{enumerate}
    \end{proof}
\end{lem}

\begin{lem}
    Seien $B_1$, $B_2$ Basen von Topologien $\tau_1$, $\tau_2$ auf $X$. Dann ist $\tau_1$
    genau dann feiner als $\tau_2$, wenn f"ur jedes $U\in B_2$ mit $x\in U$ ein $V\in B_1$
    existiert mit $x\in V\subset U$.
    \begin{proof}
    \subsubsection*{Hinrichtung}
        Sei $x\in X$, $U\in B_2$ mit $x\in U$, da $\tau_2 \subset \tau_1$ ist $U$ offen in
        $\tau_2$ (?!). Damit existiert per Definition (von was?) ein $V\in B_1$ mit
        $x\in V\subset U$.
    \subsubsection*{R"uckrichtung}
        Sei $W\in\tau_2$ und sei $x\in W$. Dann existieren $V\in B_2$ und $x\in V\subset W$.
        Damit existiert nach Annahme ein $U\in B_1$ mit $x\in U\subset V$. Also
        $X\in W \subset U$ und $W\in \tau_1$.
    \end{proof}
    \begin{bsp}
        $(X,d)$ metrischer Raum. Dann ist $\mdef[B_\eps(x)]{x\in X,\ \eps > 0}$ eine Basis
        f"ur die metrische Topologie.
    \end{bsp}
    \begin{bsp}
        $X = \R$ und $B_l = \mdef[[a,b)]{a,b\in \R,\ a < b}$. Dann ist $B_l$ eine Basis, die
        entsprechende Topologie hei\ss{}t die $blabla$ Topologie $\tau_l$.
        \begin{bem}
            $T_l$ ist strikt feiner als die euklidische Topologie.
        \end{bem}
    \end{bsp}
\end{lem}

\begin{dfn}
    Eine Teilmenge $S\subset \potmenge{X}$ ist eine \emph{Subbasis} einer Topologie auf $X$,
    falls
        \[ \forall x\in X \exists U\in S \ \text{mit}\ x\in U \].

    \begin{stz}
    \end{stz}
\end{dfn}

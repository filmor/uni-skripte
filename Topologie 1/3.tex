\section{Die Quotiententopologie}

% Tolle Zeichnungen

\begin{dfn}
    Sei $X$ eine Menge. Eine \emph{Relation} auf $X$ ist eine Teilmenge
    $R\subset X\times X$, für $(x,y)\in R$ schreibt man: $x\ R\ y$ oder $x \sim_R y$.\\
    Weitere Eigenschaften:
    \begin{enumerate}
        \item $R$ ist \emph{reflexiv}, falls $x\sim x$
        \item $R$ ist \emph{symmetrisch}, falls $x\sim y \Equiv y\sim x$
        \item $R$ ist \emph{transitiv}, falls $x\sim y,\ y\sim z \Rightarrow
            x\sim z$
        \item $R$ ist \emph{Äquivalenzrelation}, falls 1, 2 und 3 gelten.
    \end{enumerate}
\end{dfn}

\begin{dfn}
    Eine \emph{Äquivalenzklasse} von $x\in X$ ist die Teilmenge
    \[ [x] = [x]_R = \mdef[y\in X]{y \sim x} \]
    $X/R$ ist die Menge der Äquivalenzklassen. Die Abbildung $X\to X/R,\ x\mapsto[x]$
    heißt \emph{kanonische Projektion}.
\end{dfn}

\begin{dfn}
    $X$ eine Menge. Eine \emph{Partition} (\emph{Zerlegung}) von $X$ ist eine Menge 
    $\mdef{A_i}_{i\in I}$ von disjunkten Teilmengen mit $X=\bigcup_{i\in I} A_i$.
\end{dfn}

\begin{lem}
    Sei $X$ eine Menge.
    \begin{enumerate}
        \item Ist $R$ eine Äquivalenzrelation auf $X$, so ist $X/R$ eine Partition
            von $X$
        \item Ist $P$ eine Partition von $X$, so existiert genau eine
            Äquivalenzrelation $F$ mit $P=X/R$
    \end{enumerate}
\end{lem}

\begin{lem}
    Ist $f:X\to Y$ eine Abbildung, so definiert 
\end{lem}<++>

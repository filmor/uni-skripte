\section{Die Quotiententopologie}

% Tolle Zeichnungen

\begin{dfn}
    Sei $X$ eine Menge. Eine \emph{Relation} auf $X$ ist eine Teilmenge
    $R\subset X\times X$, für $(x,y)\in R$ schreibt man: $x\ R\ y$ oder $x \sim_R y$.\\
    Weitere Eigenschaften:
    \begin{enumerate}
        \item $R$ ist \emph{reflexiv}, falls $x\sim x$
        \item $R$ ist \emph{symmetrisch}, falls $x\sim y \Equiv y\sim x$
        \item $R$ ist \emph{transitiv}, falls $x\sim y,\ y\sim z \Rightarrow
            x\sim z$
        \item $R$ ist \emph{Äquivalenzrelation}, falls 1, 2 und 3 gelten.
    \end{enumerate}
\end{dfn}

\begin{dfn}
    Eine \emph{Äquivalenzklasse} von $x\in X$ ist die Teilmenge
    \[ [x] = [x]_R = \mdef[y\in X]{y \sim x} \]
    $X/R$ ist die Menge der Äquivalenzklassen. Die Abbildung $X\to X/R,\ x\mapsto[x]$
    heißt \emph{kanonische Projektion}.
\end{dfn}

\begin{dfn}
    $X$ eine Menge. Eine \emph{Partition} (\emph{Zerlegung}) von $X$ ist eine Menge 
    $\mdef{A_i}_{i\in I}$ von disjunkten Teilmengen mit $X=\bigcup_{i\in I} A_i$.
\end{dfn}

\begin{lem}
    Sei $X$ eine Menge.
    \begin{enumerate}
        \item Ist $R$ eine Äquivalenzrelation auf $X$, so ist $X/R$ eine Partition
            von $X$
        \item Ist $P$ eine Partition von $X$, so existiert genau eine
            Äquivalenzrelation $F$ mit $P=X/R$
    \end{enumerate}
\end{lem}

\begin{lem}
    Ist $f:X\to Y$ eine Abbildung, so definiert
    \begin{equation}
     x \sim_f y :\Equiv f(x) = f(z)
    \end{equation}
    eine "Aquivalenzrelation auf $X$.
    \begin{proof}
     (3.4b), Urbilder bilden Partition.
    \end{proof}
\end{lem}

\begin{lem}
 Sei $f:X \to Y$ eine Abbildung und $R$ eine "Aquivalenzrelation auf $X$ so dass
 $f(x) = f(y)$ f"ur alle $x$, $y$ mit $x \sim y$. Dann existiert eine
 \emph{einzige} Abbildung $\overline{f}: X/R \to Y$, so dass das Diagramm
 kommutiert:

 \[
 \begin{diagram}
    \node{X} \arrow{e,t}{f} \arrow{s,l}{\pi}
        \node{Y}\\
    \node{X/R} \arrow{ne,r}{\overline{f}}
 \end{diagram}
 \]

 Ist $f$ surjektiv und $R$ von $f$ wie in 3.5, so ist $\overline{f}$ bijektiv.
 \begin{proof}
  W"ahle $\overline{f}([x]) := f(x)$.
 \end{proof}
\end{lem}

\begin{lem}
 $f: X \to Y$ eine Abbildung. $\mdef{A_\alpha}_{\alpha\in I} \subset \potmenge{X}$,
 $\mdef{B_\beta}_{\beta\in J} \subset \potmenge{X}$, $D\subset Y$. Dann gilt:
 \begin{itemize}
  \item $f\left(\bigcup_\alpha A_\alpha\right) = \bigcup_\alpha f(A_\alpha)$
  \item $f\left(\bigcap_\alpha A_\alpha\right) \subset \bigcap_\alpha f(A_\alpha)$
  \item $f^{-1}\left(\bigcup_\beta B_\beta\right) = \bigcup_\beta f^{-1}(B_\beta)$
  \item $f^{-1}\left(\bigcap_\beta B_\beta\right) = \bigcap_\beta f^{-1}(B_\beta)$
  \item $f^{-1}(Y \setminus D) = X \ f^{-1}(D)$
  \item $f(f^{-1}(D)) = f(X) \cap D$
 \end{itemize}
 \begin{proof}
  Doof.
 \end{proof}
\end{lem}

\begin{dfn}
 $(X,\tau)$, $(Y, \tau')$ topologische R"aume, $f: X\to Y$ eine Abbildung. $f$
 hei"st \emph{Quotientenabbildung}, falls
 \begin{itemize}
  \item $f$ surjektiv
  \item f"ur $U \subset Y$ gilt: $U\ \text{offen} \Equiv f^{-1}(U)\ \text{offen}$
 \end{itemize}

 \begin{bem}
  \begin{itemize}
   \item Eine Quotientenabbildung ist stetig.
   \item Eine stetige surjektive Abbildung ist im Allgemeinen \emph{kein}
        Quotient. ($(X, \tau_1) \overset{id} \to (X,\tau_2)$ mit $\tau_2$ strikt
        feiner als $\tau_2$)
  \end{itemize}
 \end{bem}

 \begin{bsps}
  \begin{itemize}
   \item Ist $f:(X,\tau)\to (Y,\tau')$ stetig, surjektiv und offen oder abgeschlossen,
        dann ist $f$ ein Quotient
   \item Sei $X=[0,1]\cup [2,3] \subset \R$, $Y=[0,2]\subset \R$ mit der euklidischen
        Topologie. Dann ist $f(x)=x\chi_{[0,1]} + (x-1)\chi_{[2,3]}$ stetig,
        surjektiv und abgeschlossen, also ein Quotient
   \item Seien $X$, $Y$ topologische R"aume, dann ist $X\times Y \overset[p_X]{\to} X$
        ein Quotient
  \end{itemize}
 \end{bsps}
\end{dfn}

\begin{dfn}
  $(X,\tau)$ topologischer Raum, $Y$ eine Menge und $f:X\to Y$ eine surjektive
  Abbildung. Dann existiert genau eine Topologie $\tau_f$ auf $Y$, so dass $f$
  eine Quotientenabbildung ist. Sie ist durch $\tau_f = \mdef[U\subset Y]{f^{-1}(U)\in \tau}$
  definiert.
  \begin{proof}[Existenz]
   Lemma 37 (c) und (d)
  \end{proof}
\end{dfn}

\begin{lem}
 $(X, \tau)$, $(Y, \tau')$ topologische R"aume, $f:X\to Y$ surjektiv und $g:Y\to Z$
 eine Abbildung. Dann sind "aquivalent:
 \begin{enumerate}
  \item $g:(Y,\tau_f)\to (Z, \tau')$ ist stetig
  \item $g\circ f : (X, \tau) \to (Z,\tau')$ ist stetig
 \end{enumerate}

 \begin{proof}
  Sei $U\subset Z$. Dann gilt:
  \begin{align}
  (g\circ f)^{-1}(U)=f^{-1}(g^{-1}(U) \ \mathrm{und} \\
  g^{-1}(U)\ \mathrm{ist offen}\ \Equiv\ f^{-1}g^{-1} \ \mathrm{ist offen}
  \end{align}
 \end{proof}
\end{lem}

\begin{dfn}
 Sei $R$ eine "Aquivalenzrelation und $\tau$ eine Topologie auf $X$, $\pi$ die
 Projektion $X\to X/R$. Die Topologie $\tau_\pi$ auf $X/R$ hei"st
 \emph{Quotiententopologie}. $\pi$ hei"st \emph{kanonische Projektion}.
\end{dfn}

\begin{stz}[Universelle Eigenschaft]
 Ist $f:X\to Y$ eine stetige Abbildung, $R$ "Aquivalenzrelation auf $X$. Falls
 $x\sim y \Rightarrow f(x) = f(y)$, so gibt es genau eine Abbildung $\overline{f}$,
 so dass
 \[
  \begin{diagram}
   \node{X} \arrow{e,t}{f} \arrow{s,l}{\pi} \node{Y} \\
   \node{X/R} \arrow{ne, r}{\overline{f}}
  \end{diagram}
 \]
 kommutiert.

 \begin{bem}
  \[
   \begin{diagram}
    \node{X} \arrow{e,A,t}{f \mathrm{ surj.}} \arrow{s,A} \node{Y} \\
    \node{X/R} \arrow{ne,..,r}{\overline{f}}
   \end{diagram}
  \]
  Falls $f(x) = f(y) \Equiv x\sim y$, so ist $\overline{f}$ bijektiv aber im
  Allgemeinen \emph{kein} Hom"oomorphismus.
 \end{bem}
\end{stz}

\begin{dfn}
 $A\subset X$ Teilraum, $R_A$ die zur Partition $X=A\cup \bigcup_{x\in X\setminus A}
 \mdef{x}$ assoziierte "Aquivalenzrelation. Dann schreiben wir
 \[
  X/A := X/R_A
 \]
 \begin{bsp}
  \begin{gather}
   X = [0,1]\subset \R \\
   S^1 = \mdef[z\in\C]{\left|z\right|=1}\subset \C \cong \R^2 \\
   A = \mdef{0,1}
  \end{gather}
  Dann ist $X/A \cong S^1$. Ist $\phi:X\to Y$ ein Quotient und bijektiv, so ist
  $\phi$ ein Hom"oomorphismus. Die Einschr"ankung $\exp : [0,1) \to S^1$ ist
  bijektiv und die Einschr"ankung eines Quotienten aber nicht offen (also kein
  Hom"oomorphismus, also kein Quotient).

  \subsubsection*{Frage:} Wie vertr"aglich ist die Qutienteneigenschaft mit
  anderen Konstruktionen?
 \end{bsp}
\end{dfn}

\begin{dfn}
 Sei 
\end{dfn}

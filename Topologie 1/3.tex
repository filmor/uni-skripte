\section{Die Quotiententopologie}

% Tolle Zeichnungen

\begin{dfn}
    Sei $X$ eine Menge. Eine \emph{Relation} auf $X$ ist eine Teilmenge
    $R\subset X\times X$, für $(x,y)\in R$ schreibt man: $x\ R\ y$ oder $x
    \sim_R y$.\\ Weitere Eigenschaften:
    \begin{enumerate}
        \item $R$ ist \emph{reflexiv}, falls $x\sim x$
        \item $R$ ist \emph{symmetrisch}, falls $x\sim y \Equiv y\sim x$
        \item $R$ ist \emph{transitiv}, falls $x\sim y,\ y\sim z \Rightarrow
            x\sim z$
        \item $R$ ist \emph{Äquivalenzrelation}, falls 1, 2 und 3 gelten.
    \end{enumerate}
\end{dfn}

\begin{dfn}
    Eine \emph{Äquivalenzklasse} von $x\in X$ ist die Teilmenge \[ [x] = [x]_R
    = \mdef[y\in X]{y \sim x} \] $X/R$ ist die Menge der Äquivalenzklassen. Die
    Abbildung $X\to X/R,\ x\mapsto[x]$ heißt \emph{kanonische Projektion}.
\end{dfn}

\begin{dfn}
    $X$ eine Menge. Eine \emph{Partition} (\emph{Zerlegung}) von $X$ ist eine Menge 
    $\mdef{A_i}_{i\in I}$ von disjunkten Teilmengen mit $X=\bigcup_{i\in I} A_i$.
\end{dfn}

\begin{lem}
    Sei $X$ eine Menge.
    \begin{enumerate}
        \item Ist $R$ eine Äquivalenzrelation auf $X$, so ist $X/R$ eine Partition
            von $X$
        \item Ist $P$ eine Partition von $X$, so existiert genau eine
            Äquivalenzrelation $F$ mit $P=X/R$
    \end{enumerate}
\end{lem}

\begin{lem}
    Ist $f:X\to Y$ eine Abbildung, so definiert
    \begin{equation}
        x \sim_f y :\Equiv f(x) = f(z)
    \end{equation}
    eine "Aquivalenzrelation auf $X$.
    \begin{proof}
        (3.4b), Urbilder bilden Partition.
    \end{proof}
\end{lem}

\begin{lem}
    Sei $f:X \to Y$ eine Abbildung und $R$ eine "Aquivalenzrelation auf $X$ so dass
    $f(x) = f(y)$ f"ur alle $x$, $y$ mit $x \sim y$. Dann existiert eine
    \emph{einzige} Abbildung $\overline{f}: X/R \to Y$, so dass das Diagramm
    kommutiert:

    Ist $f$ surjektiv und $R$ von $f$ wie in 3.5, so ist $\overline{f}$ bijektiv.
    \begin{proof}
        W"ahle $\overline{f}([x]) := f(x)$.
    \end{proof}
\end{lem}

\begin{lem}
    $f: X \to Y$ eine Abbildung. $\mdef{A_\alpha}_{\alpha\in I} \subset
    \potmenge{X}$, $\mdef{B_\beta}_{\beta\in J} \subset \potmenge{X}$,
    $D\subset Y$. Dann gilt:
    \begin{itemize}
        \item $f\left(\bigcup_\alpha A_\alpha\right) = \bigcup_\alpha f(A_\alpha)$
        \item $f\left(\bigcap_\alpha A_\alpha\right) \subset \bigcap_\alpha
            f(A_\alpha)$
        \item $f^{-1}\left(\bigcup_\beta B_\beta\right) = \bigcup_\beta
            f^{-1}(B_\beta)$
        \item $f^{-1}\left(\bigcap_\beta B_\beta\right) = \bigcap_\beta
            f^{-1}(B_\beta)$
        \item $f^{-1}(Y \setminus D) = X \ f^{-1}(D)$
        \item $f(f^{-1}(D)) = f(X) \cap D$
    \end{itemize}
    \begin{proof}
        Doof.
    \end{proof}
\end{lem}

\begin{dfn}
    $(X,\tau)$, $(Y, \tau')$ topologische R"aume, $f: X\to Y$ eine Abbildung. $f$
    hei"st \emph{Quotientenabbildung}, falls
    \begin{itemize}
        \item $f$ surjektiv
        \item f"ur $U \subset Y$ gilt: $U\ \mathrm{offen} \Equiv f^{-1}(U)\
            \mathrm{offen}$
    \end{itemize}

    \begin{bem}
        \begin{itemize}
            \item Eine Quotientenabbildung ist stetig.
            \item Eine stetige surjektive Abbildung ist im Allgemeinen
                \emph{kein} Quotient. ($(X, \tau_1) \overset{id} \to
                (X,\tau_2)$ mit $\tau_2$ strikt feiner als $\tau_2$)
        \end{itemize}
    \end{bem}

    \begin{bsps}
        \begin{itemize}
            \item Ist $f:(X,\tau)\to (Y,\tau')$ stetig, surjektiv und offen
                oder abgeschlossen, dann ist $f$ ein Quotient
            \item Sei $X=[0,1]\cup [2.3] \subset \R$, $Y=[0,2]\subset \R$ mit
                der euklidischen Topologie. Dann ist $f(x)=\choose$
        \end{itemize}
    \end{bsps}
\end{dfn}

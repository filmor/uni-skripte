\documentclass{skript}

\begin{document}

\title{Geometrie I}
\author{}
\maketitle
\tableofcontents
\renewcommand{\d}{\ensuremath{\mathrm d}}

\section{Untermannigfaltigkeiten in euklidischen Räumen}

\begin{dfn}
Sei $f : M \to \R^n$ (wobei $M$ eine Mannigfaltigkeit ist) eine
\emph{Immersion}, das heißt $df(p) : T_p M \to \R^n$ (das Differential als
Abbildung aus dem Tangentialraum nach $\R^n$) ist injektiv für alle $p\in M$.
Ist $n=1$ also $f$ eine Kurve, dann ist dieser Begriff gleichbedeutend damit,
dass die Kurve \emph{regulär} ist.

\begin{bsps}
  \begin{enumerate}
    \item[Graphen] Sei $g: U \to \R^k$ glatt mit $f:U\to\R^n$ und $n = m + k$,
      $U\subset \R^m$ offen. $g$ ist definiert durch $f(p) = (p,g(p))$
      % Bild: U in der xy-Ebene, f(U) darüber, g(p) Höhe

      Dann gilt $\d f(p)(v) = (v, \d g(p)(v))$, also ist $\d f(p)$ injektiv.
    \item[Drehflächen] (schon diskutiert)
    \item Sei $M\subset\R^n$ Untermannigfaltigkeit. Dann ist die Inklusion
      $\iota : M \to \R^n$ eine Immersion.
  \end{enumerate}
\end{bsps}
\end{dfn}

Sei $I = [a,b]$ und $c : I \to M$ glatt. Dann ist $f\circ c : I \to \R^n$ glatt
der Länge
\begin{align}
  L(f\circ c) &= \int_a^b\left|(f\circ c)'(t)\right| \d t \\
              &= \int_a^b\left|\d f(c(t))(c'(t))\right| \d t
\end{align}

Wir möchten nun diese Kurvenlänge schon in $M$ bestimmen können. Dazu erklären
wir auf $T_pM$ für alle $p\in M$ folgendes Skalarprodukt:
\begin{align}
  g_p(v, w) := \sprod[p]{v,w} = \sprod{\d f(p)\cdot v, \d f(p) \cdot w}
    \quad v,w \in T_pM
\end{align}
Die zugehörige Norm bezeichnen wir mit $\norm[p]{\cdot}$, also
$\norm[p]{v} = \norm{\d f(p)\cdot v}, v\in T_p M$. Damit ist die
Länge der Kurve:
\begin{align}
  L(f\circ c) &= \int_a^b\norm{\d f(c(t))\cdot c'(t)} \d t
              &= \int_a^b\norm[c(t)]{c'(t)} \d t
\end{align}

Die innere Geometrie von $M$ wird also durch die Familie $g_p$ von
Skalarprodukten bestimmt.


Sei nun $U\subset M$ offen und $x : U \to U' \subset \R^m$ eine Karte für $M$
über $U$. Zugeordnet zu dieser Karte haben wir die Koordinatenvektorfelder
\begin{align}
  X_j = \frac{\partial}{\partial x^j}, \quad ^\leq j \leq m
\end{align}

Dann ist $(X_1(p), \dots, X_m(p))$ für alle $p\in M$ eine Basis von $T_p M$.

Wir definieren Funktionen $g_{ij} : U\to \R$ durch
\begin{align}
  g_{ij}(p) &:= \sprod[p]{X_i(p), X_j(p)} \\
            &= \sprod{\d f X_i(p), \d f X_j(p)} \\
            &= \sprod{\frac{\partial}{\partial x^i(p)},
            \frac{\partial}{\partial x^j(p)}} \\
            &= \sprod{f_i(p), f_j(p)}
\end{align}
Damit sehen wir, dass die $g_{ij}$ glatt sind.

Seien $X,Y$ Vektorfelder auf $U$,
\begin{align}
  X(p) = \xi^i(p)X_i(p), \qquad Y(p) = \eta^i(p)X_i(p)
\end{align} (Es gilt die Einsteinsche Summenkonvention, hier wird also über $i$
summiert.)

Nach den Definitionen sind $X$ und $Y$ glatt genau dann, wenn ihre
\emph{Hauptteile} $\xi, \eta$ bezüglich x glatt sind.

Wir erhalten
\begin{align}
  \sprod[p]{X(p), Y(p)} &= \sprod[p]{\xi^i(p)X_i(p), \eta^j(p)X_j(p)} \\
                        &= \xi^i\eta^jg_{ij}
\end{align}

Wir haben also das Skalarprodukt bezüglich des Differentials zurückgezogen.

Wir nennen die Familie $g = (g(p))_{p\in M}$ von Skalarprodukten die \emph{erste
Fundamentalform} von $f$.


\begin{dfn}[Lie-Gruppe]
  Eine \emph{Lie-Gruppe} ist eine Gruppe mit einer differenzierbaren Struktur,
  \begin{align}
    g\mapsto g^{-1}, \quad (h,g)\mapsto h\cdot g
  \end{align}
  $C^\infty$-Abbildungen sind.

  \begin{bsps}
    \begin{itemize}
      \item $\mdef[A\in \R^{n\times m}]{A^tBA=B}$
      \item $O, Gl, SO, Sl, U$
    \end{itemize}
  \end{bsps}
\end{dfn}


\begin{dfn}
   Eine differenzierbare Abbildung $f : M \to N$, $M,N$ differenzierbaren
   Mannigfaltigkeiten, heißt \emph{Submersion} falls $\d f : T_x M \to
   T_{f(x)}N$ surjektiv ist.
\end{dfn}

\begin{dfn}
  Sei $G$ eine Gruppe. Eine (linke) \emph{Aktion} ist eine Abbildung $\phi:
  G\times M \to M$ ($M$ Mannigfaltigkeit) sodass
  \begin{itemize}
    \item $\varphi(g_1, \varphi(g_2, m)) = \phi(g_1g_2, m)$
    \item $\varphi(e, m) = m$
  \end{itemize}

  $\varphi$ heißt \emph{frei}, falls aus $gm = m$ immer $g = e$ für alle $m\in
  M$ folgt.

  $\varphi$ heißt \emph{eigentlich}, falls für alle kompakten Mengen
  $K\subset M$ die Menge $\mdef[g\in G]{gK\cap K \neq \emptyset}$ einen
  kompakten Abschluss hat.

  $\varphi$ heißt \emph{transitiv}, falls für alle $m_1, m_2 \in M$ ein $s\in G$
  mit $gm_1 = m_2$ existiert. Der Quotient $M/G = \mdef[\mdef[gm]{g\in G}]{m\in
  M}$ ist die Menge der Orbits.
\end{dfn}

\begin{stz}
  $M$ sei eine Mannigfaltigkeit, $G$ eine Lie-Gruppe die glatt auf $M$ wirkt. Es
  gibt eine differenzierbare Struktur auf $M/G$ genau dann, wenn die Aktion frei
  und eigentlich ist.
  \begin{proof}
    Gibt's hier nicht :/
  \end{proof}
\end{stz}


\begin{stz}
  Die folgenden Aussagen sind äquivalent:
  \begin{itemize}
    \item Die Lie-Gruppe $G$ wirkt transitiv und glatt auf $M$, mit $H =
      \mdef[h\in H]{hm = m}$
    \item $M$ ist diffeomorph zu $G/H$
  \end{itemize}
\end{stz}

\begin{bsp}
  Sei $p\leq n$, $S_{n,p} = \mdef[(x^k)_{k=1\dots p}]{}$
  
\end{bsp}
Sei $f : M \to \R^n$ eine Immersion. $g$ ist die \emph{erste Fundamentalform} zu
$F$\dots

$v,w \in T_p M : \sprod[p]{v,w} := \sprod{\d f(p)v, \d f(p)w}$. Falls
$(x, U)$ eine Karte mit zugehörigen Vektorfeldern
\begin{align}
  X_1 = \frac{\partial}{\partial x^1}, \dots, X_m = \frac{\partial}{\partial x^m}
\end{align}
Dann ist $g_{ij}(p) := \langle X_i(p), X_j(p)\rangle_p, p\in U$ die
\emph{Fundamentalmatrix} von $g$ bezüglich $X_1, \dots, X_m$.

Falls $X = \xi^iX_i, Y = \eta^iX_i$ Vektorfelder von $M$ über U sind, dann ist
$\langle X, Y\rangle = \xi^i\eta^jg_{ij} : U \to \R$.

Falls $c : I \to U \subset M$ glatt, so ist
\begin{align*}
  L(f\circ c) &= \int_I\norm{(f\circ c)'(t)}\d t \\
              &= \int_I\norm{\d f(c(t)) \cdot c'(t)}\d t \\
              &= \int_I\norm[c(t)]{c'(t)} \d t \\
              &= \int_I\norm[c(t)]{(c^i)'(t)X_i(c(t))} \d t \\
              &= \int_I\left( (c^i)'(t)(c^j)'(t)g_{ij}(c(t))
              \right)^{\frac{1}{2}}\d t \\
              &=: L_g(c)
\end{align*}
die Länge von $c$ bezüglich $g$.

Zu $p,q\in M$ setzen wir
\begin{align}
  d(p,q) &= d_g(p,q) &= \inf\mdef[L(c)]{c:[a,b]\to M \text{ ist stückweise glatt
  mit } c(a) = p \text{ und  } c(b) = q}
\end{align}
Wir definieren also den Abstand als das Infimum der Länge aller glatten Wege von
$p$ nach $q$ in der Mannigfaltigkeit.

\begin{stz}
  $d$ ist eine Metrik auf $M$, die die auf $M$ vorgegebene Topologie induziert.
  \begin{proof}
    Symmetrie und $\triangle$-Ungleichung sind klar. Für alle $p,q \in M$ ist
    $d(p,q)\geq 0$ (per def.).

    Seien $p,q\in M, p\neq q$. Wähle eine Karte $(x, U)$ von $p$ mit $q\notin U$
    und $x(p) = 0$. Nun sind die $g_{ij} : U \to \R$ (wie oben) glatt und
    $\left( g_{ij}(p) \right)$ ist positiv definit $\forall p\in U$.

    Wähle $\delta > 0$ so, dass $\mdef[u\in \R^n]{\norm{u} \leq \delta} = \bar
    B_\delta (0) \subset \R^m$ enthalten ist in $x(U) =: U'$. Zu diesem $\delta$
    gibt es dann ein $\epsilon > 0$ mit
    \begin{align}
      \epsilon^2 \delta_{ij}v^iv^j &\leq g_{ij}(p')v^iv^j \\
      &\leq \frac{1}{\epsilon^2} \delta_{ij}v^iv^j
      \label{eqn:5.g_ungleichung}
    \end{align}
    für alle $v = (v^1,\dots,v^m)\in \R^m$ und alle $p'\in U$ mit $x(p')\in \bar
    B_\delta (0)$.

    Falls nun $c : [a, b] \to M$ eine stückweise glatte Kurve von $p$ nach $q$
    ist, $c(a) = p$ und $c(b) = q$, dann gibt es ein kleinstes $t_0 > a$ mit
    $\norm{x\circ c)(t)}$, $0 \leq t < t_0$ und $\norm{(x\circ c)(t_0)} =
    \delta$. Daher ist
    \begin{align}
      L_g(c) &\ge L_g(c\vert_{[a,b]}) = \int_0^{t_0} \sqrt{g_{ij}(c(t))(x^i\circ
      c)'(t)(x^j\circ c)'(t)}\d t \\
             &\ge \epsilon \int_0^{t_0}
             \sqrt{\sum_{i=1}{n}(x^i\circ c)'(t)(x^i\circ c)'(t)}\d t \\ &= \epsilon
             L\left( (x\circ c)\vert_{[a,t_0]} \right) \ge \epsilon\delta > 0
             \label{eqn:5.lg_ungleichung}
    \end{align}
    Dies folgt aus Ungleichung \ref{eqn:5.g_ungleichung} und daraus, dass
    $(x\circ c)\vert_{[a,t_0]}$ ein Weg ist, der $0$ mit einem Punkt auf der
    $\delta$-Sphäre verbindet.

    Damit ist $d_g(p,q) \ge \epsilon\delta$, also ist $d_g$ eine Metrik.


    Sei nun $V \subset M$ offen und $p\in V$. Wähle Karte $(x, U)$ um $p$ mit
    $x(p)$ und $U \subset V$. Wähle $\epsilon$ und $\delta$ wie oben. Falls dann
    $q\in M$ mit $q\notin U$ (oder $q\in U$ mit $\norm{x(q)}\ge 1$), so ist
    $d(p,q) \ge \epsilon\delta$ mit dem selben Argument wie oben. Daher ist
    $\mdef[q\in M]{d_g(p,q)<\epsilon\delta}\subset U \subset V$.

    Sei nun umgekehrt $p\in M$ und
    \begin{align}
      B = \mdef[q\in M]{d_g(p,q) < \alpha}
      \label{}
    \end{align}
    für ein $\alpha > 0$. Dann enthält $B$ eine Umgebung (bezüglich der
    vorgegebenen Topologie) von $p$. Wähle dazu $(x, U)$, $\epsilon$ und
    $\delta$ wie oben. Sei $u = x(q)$ mit $\norm u \ge \delta$ und sei $c(t) =
    x^{-1}(tu)$, $0 \le t \le 1$. Nach Ungleichung \ref{eqn:5.lg_ungleichung}
    ist $L_g(c) = \int_0^1 \sqrt{g_{ij}(c(t))u^iu^j}\d t \le
    \frac{\norm u}{\epsilon}$. Daher ist $\mdef[q\in U]{\norm{x(q)} <
    \epsilon\alpha} \subset B$.
  \end{proof}

  \begin{bsps}
    \begin{enumerate}
      \item $\id : \R^m \to \R^m$
      \item $S^m = \mdef[x\in \R^{m+1}]{\norm x = 1}$, $f : S^m\to\R^{m+1}$ die
        Inklusion.
        
        \emph{Behauptung:} $d(p,q) = \sphericalangle(p,q)$

        Sei dazu $p\in S^m$. Parametrisiere Punkte in $S^m$ durch $q = (\cos t)p
        + (\sin t)y, y\in p^\perp, \norm y = 1$. Für $0< t < \pi$ ist dann $q\in
        S^m \setminus \mdef{\pm p}$. Auf dieser Teilmenge erklären wir
        Vektorfelder $V$ durch $V(q) = -(\sin t)p + (\cos t)y$. Es ist $\norm
        V(q) = 1$. Sei nun $c:[a,b] \to S^m$ eine stückweise glatte Kurve von
        $p$ nach $q$.
        oBdA ist $c([a,b])\subset S^m\setminus\mdef{\pm p}$ und 

        % Es wird gezeigt, dass d(p,q) immer größer gleich ist als der Winkel
        % zwischen p und q. Wählt man als d(p,q) die Länge des Großkreisstückes,
        % das p und q enthält erhält man Gleichheit.
    \end{enumerate}
  \end{bsps}
\end{stz}

% 28.10.2009
Zu $p\in M$ setzen wir $T_pf = \im{\d f(p)} \subset \R^n$, ein linearer
Unterraum, interpretiert als tangentiale Komponente an das Bild von M unter $f$
im Punkt $p$. Das orthogonale Komplement von $T_pf$ in $\R^n$ bezeichnen wir mit
$N_pf$, dem \emph{Normalraum} von $M$ in $p$ bezüglich $f$.

\begin{bsp}
  $M=S^m$ mit $f:S^m\to \R^{m+1}$ die Inklusion. Dann ist
  \begin{align*}
    T_pf &= \mdef[v\in\R^{m+1}]{\sprod{p,v}=0},\\
    N_pf &= \mdef[v\in\R^{m+1}]{\exists \lambda\in\R : v = \lambda p}
  \end{align*}
\end{bsp}

\begin{lem}
  Zu jedem Punkt $p\in M$ gibt es eine offene Umgebung $U$ in $M$ und (glatte)
  Vektorfelder $X_1, \dots, X_n : U\to \R^n$ längs $f$ (das heißt $X_j$ ist eine
  glatte Abbildung wobei $X_j(q)$ als Tangentialvektor des $\R^n$ in $f(q)$
  interpretiert wird), für die gilt
  \begin{align*}
    X_1(q), \dots, X_m(q) \quad & \text{sind Orthonormalbasis von } T_qf \\
    X_{m+1}(q), \dots, X_n(q) \quad & \text{sind Orthonormalbasis von } N_qf \\
    \forall q\in U
  \end{align*}
  \begin{proof}
    Sei $(U_0, x)$ eine Karte von $M$ mit $p\in U_0$. Wähle eine Basis
    $v_{m+1},\dots,v_n$ von $N_pf$. Dann ist $\frac{\partial f}{\partial
    x^1}(p),\dots,\frac{\partial f}{\partial x^m}(p), v_{m+1},\dots,v_n$ eine
    Basis des $\R^n$, diese Vektoren sind also linear unabhängig.

    % Bild: Kurve die f(M) darstellt als Linie mit den (linearen) Normal- und
    % Tangentialräumen

    Die Menge aller $q\in U_0$ so dass diese Vektoren linear unabhängig sind ist
    offen und enthält $p$. Dies sieht man, indem man die Vektoren als Spalten in
    eine Matrix schreibt und die Determinante dieser Matrix betrachtet. Da
    $\operatorname{det}$ stetig ist und die Spaltenfunktionen glatt sind, ist
    das Urbild der offenen Menge $\R\setminus\mdef{0}$ wieder offen.
    % Warum ist p enthalten?

    Auf diese Basis in der sich ergebenden offenen Menge wendet man nun das
    Gram-Schmidt-Verfahren an und erhält so eine Orthonormalbasis.
    % Warum erhält dies N_pf und T_pf?
  \end{proof}
  Insgesamt bilden also die $X_1(q),\dots, X_n(q)$ eine Orthonormalbasis von
  $\R^n$ für alle $q\in U$.

  \begin{krl}
    Sei $X:M\to \R^n$ glatt (aufgefasst als Vektorfeld längs $f$). Zu $p\in M$
    sei $X^\top(p)$ die tangentiale und $X^\perp(p)$ die normale Komponente
    von $X(p)$, also $X^\top(p)\in T_pf$ und $X^\perp(p)\in N_pf$. Dann
    sind $X^\top,X^\perp: U\to\R^n$ glatt.

    \begin{proof}
      Wähle $U$ und $X_1,\dots,X_n$ wie im Lemma. Dann ist
      $X^\top=\sum_{i=1}^m\sprod{X,X_i}X_i$ und $X^\perp = X- X^\top$      
    \end{proof}
    Dies zeigt, dass die Regularität eine lokale Eigenschaft ist.
  \end{krl}
\end{lem}

\subsection{Erste Variation der Bogenlänge}

Sei $c:[a,b]\to M$ eine stückweise glatte Kurve. Eine \emph{stückweise glatte
Variation} von $c$ ist eine stetige Abbildung $H:I\times [a,b]\to M$, $I =
(-\eps,\eps)$, so dass eine Unterteilung $t_0 = a < t_1 < \dots < t_k = b$ von
$[a,b]$ existiert, für die gilt, dass $H\vert_{I\times[t_{i-1},t_i]}\to M$ für
alle $1 \le i \le k$ glatt ist und $c = H(0,\cdot)$ ist. Zu $H$ setzen wir auch
$c_s = H(s,\cdot), \ -\eps<s<\eps$. Dann ist also $c = c_0$

Eine Variation ist also eine kleine Abweichung nach "`oben"' und "`unten"' von
der Kurve.

% Bild: Stück aus \R^2 (-\eps,\eps als Höhe, 0 in der Mitte; x-Achse unterteilt
% in t_1, \ldots, t_k) geht unter H auf eine gekrümmte Fläche mit c in der Mitte


Eine solche Variation (von c) nennen wir \emph{eigentlich}, wenn $c_s(a) = c(a)$
und $c_s(b) = c(b)$ ist für alle $s\in(-\eps,\eps)$. Sie weicht dann also an den
Endpunkten nicht ab.

Was uns nun interessiert ist die Abbildung $s\mapsto L(c_s)$, insbesondere ob
diese einen kritischen Punkt in $s=0$ besitzt, da die Kurve dann eine
Extremalkurve ist (also kürzeste oder längste Verbindung zwischen den beiden
Punkten).

Zunächst betrachten wir noch mal die Definition der Länge einer Kurve:
\begin{align*}
  L(c_s) &= \int_a^b\norm[c(t)]{c_s'(t)}\d t \\
  &= \int_a^b\sprod[c_s(t)]{c_s'(t),c_s'(t)}^{\frac{1}{2}} 
\end{align*}
Wir nehmen jetzt an, dass $\norm{c'(t)}$ konstant und $\neq 0$ ist. Dann ist
$c_s'(t) \neq 0$ für alle $t\in[a,b]$ und alle $s\in(-\eps,\eps)$ nahe an $s=0$.

Wir rechnen:
\begin{align*}
  \frac{\d L(c_s)}{\d s}\vert_{s=0} &= \frac{\d}{\d
  s}\vert_{s=0}\int_a^b\sprod[c_s]{c_s', c_s'}^{\frac{1}{2}} \\
  &= \int_a^b\frac{\d}{\d s}\sprod[H(0,t)]{\frac{\partial H}{\partial
  t},\frac{\partial H}{\partial t}}^{\frac{1}{2}}(0,t) \\ &=
  \int_a^b\frac{\sprod{\frac{\partial^2 H}{\partial s\partial t},\frac{\partial
  H}{\partial t}}}{\sprod{\frac{\partial H}{\partial t},\frac{\partial
  H}{\partial t}}^{\frac{1}{2}}}(0,t) & \frac{\partial H}{\partial t}(0,t) =
  c'(t) \\
  &= \frac{1}{\norm{c'}}\int_a^b \sprod{\frac{\partial^2 H}{\partial s\partial
  t},\frac{\partial H}{\partial t}}(0,t) \\
  &= \frac{1}{\norm{c'}}\int_a^b\left(\frac{\d}{\d t}\sprod{\frac{\partial
  H}{\partial s},\frac{\partial H}{\partial t}} - \sprod{\frac{\partial
  H}{\partial s},\frac{\partial^2 H}{\partial t^2}}\right)
\end{align*}
\begin{align*}
  \frac{\d}{\d s}L(c_s)\vert_{s=0} \\
    &= \int_a^b\frac{\d}{\d s} \norm[c_s']{c_s'}\d t\vert_{s=0} \\
    &= \int_a ^b \frac{\d}{\d s}\sprod[c_s']{c_s', c_s'}^\frac12
       \d t\vert_{s=0} \int_a^b \frac{\d}{\d s}
       \sprod{\d f(c_s(t))c_s'(t), \d f(c_s(t))c_s(t)}^\frac12\d t
       \vert_{s=0} \\
    &= \int_a^b\frac{\d}{\d s}\sprod{(f\circ c_s)'(t),(f \circ c_s)'(t)}
       \frac12\d t\vert_{s=0} \\
  %  &= \dots = \frac1\norm[c']{\sprod{V(b),c'(b)}-sprod{V(a),C(a)}}
  %     + \sum_{i=1}^k-1 \sprod{V(t_i),c'(t_i-)-c'(t_i+)}
  %     + \int_a^b \sprod{(f\circ c)''(t), \d f(c(t))V(t)}
\end{align*}
Wir setzen jetzt $V(t) := \frac{\partial H}{\partial s}(0,t)$, das
Variationsfeld. Außerdem ist $\frac{\partial H}{\partial t}(0,t) = c'(t)$,
$\frac{\partial^2 H}{\partial t^2}(0,t) = c''(t)$. Also ist 
\begin{align*}
  \frac{\d L(c_s)}{\d s}\vert_{s=0}
    &= \frac1{\norm{c'}}\int_a^b\left(\frac{\d}{\d
  t}\sprod{V,c'}-\sprod{V,c''}\right) \\
  &= \frac{1}{\norm{c'}}\sum_{i=1}^k\sprod{V(t),c'(t)}\vert_{t=t_{i-1}}^{t=t_i}
  - \frac{1}{\norm{c'}}\int_a^b\sprod{V,c''} \\
  &= \frac{1}{\norm{c'}}\left( \sprod{V(b),c'(b)} - \sprod{V(a),c'(a)} +
  \sum_{i=1}^{k-1}\sprod{V(t_i), c'(t_i-)- c'(t_i+)}  \right) \\
  &= - \frac{1}{\norm{c'}}\int_a^b\sprod{V,c''}
\end{align*}
Dabei sind $c'(t_i\pm)$ die links- und rechtsseitigen Limiten. Ist $c$ an dieser
Stelle stetig differenzierbar, so fallen diese Terme weg.

Ist die Variation eigentlich (wie oben definiert), so ist $V(a) = 0$ und $V(b) =
0$.

Wir sagen, dass $c:[a,b]\to M$ eine \emph{Geodätische} ist, wenn $c$ glatt mit
konstanter Geschwindigkeit ($=$ Norm der ersten Ableitung) ist und
$\frac{\d L(c_s)}{\d s}\vert_{s=0}$ für alle eigentlichen Variationen von $c$
gilt.

Um Geodätische zu diskutieren nehmen wir zunächst einmal an, dass $c:[a,b]\to M$
stückweise glatt mit konstanter Geschwindigkeit $\norm{c'}\neq 0$ ist und dass
$\frac{\d L(c_s)}{\d s}\vert_{s=0}$ für alle eigentlichen Variationen von $c$
gilt.

Zunächst bemerken wir dazu, dass $V(t)\in T_{c(t)}M$ ist, $V(t) =
\frac{\partial H}{\partial s}(0,t)$. Zu vorgegebenen $v_0\in T_{c(t)}M$, $t\in
(t_{i-1},t_i)$, gibt es eine eigentliche Variation $H$ von $c$ mit
Variationsfeld $V$, so dass $V(\tau) = 0$ für alle $\tau\in[a,b]\setminus
(t_{i-1},t_i)$, sonst $V(t) = v_0$.

% Bild: Lange Kurve mit den Endpunkten c(a) und c(b), dazwischen kurzes
% Teilstück c(t_{i-1}) -> c(t_{i}). Zwischen diesen beiden Punkten ist c(t)
% markiert, von dem aus ein Pfeil nach außen geht der mit v_0 bezeichnet ist.

Wähle eine Kurve $\sigma = \sigma(s)$, $-\eps<s<\eps$, in $M$ mit $\sigma(0) =
c(t)$ und $\dot\sigma(0) = v_0$. Wähle eine Karte $(U,x)$ von $c(t)$ mit
$x(c(t))=0$. Wähle eine glatte Funktion $\phi=\phi(\tau)$ mit $\phi(t) = 1$ und
$\phi(\tau) = 0$ für $\tau \le t- \delta$, $\tau \ge t + \delta$ und setze
\begin{align*}
  H(s,\tau) = \begin{cases}
    c(\tau) & a \le \tau \le t- \delta, t+\delta\le\tau\le b \\
    x^{-1}(x(c(\tau)) + \phi(\tau x(\sigma(s)))) & \text{sonst}
  \end{cases}
\end{align*}

% Bilder, Variation nur in dem kleinen Teilstück von obigen Bild

$c:[a,b] \to M$ stückweise glatt, $\norm{c'}=const\neq0$ mit $
\frac{\d L(c_s)}{\d s}\vert_{s=0}=0$ für alle eigentichen Variationen $c_ss$ von
$c=c_0$.
%Bild Eigentliche Variation von Kurven
$c_s(a)=c(a)$, $cs(b)=c(b)$ $\forall s$
\emph{Behauptung:} Dann ist $c$ glatt und $(f\circ c)''$ ist normal zu

%Da fehlt was
Damit gilt für alle eigentlichen Variationen von c:
$\frac{\d L(c_s)}{\d t} =
\frac{1}{\norm[c']{\sum_{i=1}^{k-1}\sprod{V(t_i),c'(t_i-)-c'(t_i+)}}}$ Für alle
$t \in (a,b)$ ist $c'(t+) = c'(t-)$. Zum Beweis nehmen wir einmal an $u =
C'(t-)-c'(t+)\neq 0$.  Wähle Koordinaten $x$ um $c(t)$ und schreibe $u=u^i
\frac{\partial t}{\partial x^i}(c(t))$. Wähle $\delta>0$, sodass $c$ im
intervall $[t-s\delta,t+2\delta]$ außer bei $t=\tau$ keinen weiteren Knick hat.
Für $\phi$ wie oben und $\eps>0$ klein genug, setze $H(s,t)=\begin{cases}c(\tau)
  u\leq\tau\leq t-s\delta \end{cases}$

Damit folgt $c'(t-) = c'(t+) \forall t\in(a,b)$. In diesem Sinne ist $c$ glatt
($C^1$).
\begin{bsps}
  \begin{enumerate}
    \item Sei $S^m(r) = \mdef[x\in\R^{m+1}]{\norm{x} = r}$ wobei $r > 0$
      vorgegeben ist, $f$ die Inklusion. Für alle $x\in S^m(r)$ und $v\perp x$
      mit $\norm{v} = r$ ist $\cos t \cdot x + \sin t \cdot v =: c(t)$, $t\in
      \R$ Geodätische mit $c(0) = x$, $c'(0) = v$. Ferner ist $c''(t) = -\cos
      t\cdot x - \sin t \cdot v$, also ist $c''(t)\in N_{c(t)}f,
      f:S^m\to\R^{m+1}$ die Inklusion $\forall t\in\R$. Also ist $c$ eine
      Geodätische.
    \item Sei $(r,h):I\to\R^2$ glatt und nach Bogenlänge parametrisiert. Dann
      ist $f:I\times\R\to\R^3$ mit $f(t,\phi)= (r(t)\cos\phi, r(t)\sin\phi,
      h(t))$. Für festes $\phi_0$ ist $c:I\to\R^3, c(t) := f(t,\phi_0)$ eine
      Geodätische:
      \begin{align*}
        c'(t) &= (r'(t)\cos\phi,r'(t)\sin\phi,h'(t)) \\
        c''(t) &= (r''(t)\cos\phi,r''(t)\sin\phi,h''(t)) \\
        T_{c(t)}f &= \sprod{\mdef{\frac{\partial f}{\partial t}(t,\phi_0),
        \frac{\partial f}{\partial \phi}(t,\phi_0)}} \\
        &= \sprod{\mdef{c'(t), (-r(t)\sin\phi_0, r(t)\cos\phi_0, 0)}}
      \end{align*}
      Nun ist $\sprod{c'(t), c''(t)} = 0$, denn $\sprod{c',c'} = \mathrm{const}
      = 1$ und $\sprod{c''(t),\frac{\partial f}{\partial\phi}(t,\phi_0)} = 0$,
      also ist $c$ Geodätische.
  \end{enumerate}
\end{bsps}

\subsection{Die geodätische Differentialgleichung}

Sei $c: I\to M$ eine glatte Kurve. Sei $t_0\in I $ und $x:U\to U'\subset \R^m$
eine Karte von $M$ um $c(t_0)$. Nach eventueller Verkleinerung von I nehmen wir
a, dass $c(I) \subset U'$.
% Bild: Menge U mit einer Kurve darin und dem Punkt c(t_0) in der Mitte,
%       wird durch f auf \R^n abgebeildet und durch x auf U' \subset \R^m

Dann ist $f\circ c = (f\circ x^{-1})\circ(x\circ c)$ und damit
\begin{align*}
  (f\circ c)'(t) &= \frac{\partial f}{\partial x^i}(c(t))\cdot
  \frac{\d(x^i\circ c)}{\d t}(t) \\
  &= f_i(c(t))\frac{\d c^i}{\d t}(t), & f_i = \frac{\partial f}{\partial x^i},
  c^i = x^i\circ c..
\end{align*}
Kürzer geschrieben ist das $(f\circ c)' = f_i\cdot\frac{\d c^i}{\d t}$.

\subsubsection{Zweite Ableitung}
\begin{align*}
  (f\circ c)'' = f_{ij}\frac{\d c^j}{\d t}\frac{\d c^endi}{\d t} + f_i \cdot
  \frac{\d^2c}{\d t^2}
\end{align*}
mit $f_{ij} = \frac{\partial^2 f}{\partial x_i\partial x_j}$. Der tangentiale
Anteil ist
\begin{align*}
  T_{c(t)}f \ni \left( (f\circ c)''(t) \right)^\top &=
  \left( f_{ij}(c(t))\frac{\d c^i}{\d t}(t)\frac{\d c^j}{\d t}(t) +
  f_i(c(t)\frac{\d^2 c}{\d t^2}(t) \right)^\top \\
  &= f_{ij}^\top(c(t)) \frac{\d c^i}{\d t}(t) \frac{\d c^j}{\d t}(t) + f_i(c(t))
  \frac{\d^2 c^i}{\d t^2}(t)
\end{align*}

Weil die $f_i(p)$, $1\le i\le m$, $p\in U$, eine Basis von $T_pf$ sind, gibt es
Skalare $\Gamma^k_{ij}$ mit $f_{ij}(p) = \Gamma^k_{ij}(p)f_k(p)$. Nach einem
schon bewiesenen Lemma ist $f_{ij}^\top:U\to\R^n$ glatt, also sind auch die
$\Gamma^k_{ij}:U\to\R$ glatt. Man nennt die $\Gamma_{ij}^k$
\emph{Christoffelsymbole}.

Damit erhalten wir die \emph{geodätische Differentialgleichung} bezüglich der
Karte $x$:
\begin{align*}
  \frac{\d^2c^k}{\d t^2} + \Gamma^k_{ij}\frac{\d c^i}{\d t}(t)
  \frac{\d c^j}{\d t}(t) = 0
\end{align*}

Aus der Theorie der gewöhnlichen Differentialgleichungen kommt unter anderem:
\begin{enumerate}
  \item Zu gegebenem $v \in T_pM$ gibt es eine Geodätische $c$ mit $c'(0)=v$.
    (Insbesondere ist $c(o)$ der Fußpunkt von $v$). Falls $c_1: I_1 \to M, c_2:
    I_2\to M$ mit $t_0 \in I_1\cup I_2$ mit $c_1'(t_0)=c_2'(t_0)$, dann ist
    $c_1=c_2$ auf $I_1\cup I_2$. Damit gibt es eindeutige und maximale Geodätische
    $c_0:I\to M$ mit $0\in I$ und $c_0'(0)=v$.
  \item Falls $c: I \to N$ Geodätische, so ist auch $\tilde{c}=c(at+b)$, 
    $\mdef[t\in\R]{at+b\in I}$ eine.
\end{enumerate}

\subsection{Christoffelsymbole}
% Bild Teilmenge U einer Mfk geht unter x auf U'\subset \R^m, f \to \R^n
Die $\Gamma_{ij}^k:U\to\R$ sind glatt. Betrachte
\begin{align*}
  \sprod[p]{\Gamma^k_{ij}\frac{\partial}{\partial x^k},\frac{\partial}{\partial
  x^l}} &= \Gamma_{ij}^k\sprod[p]{\frac{\partial}{\partial
  x^k},\frac{\partial}{\partial x^l}} \\
  &= \Gamma_{ij}^kg_{kl} 
\end{align*}, wobei $\sprod[p]{\cdot,\cdot}$ die erste Fundamentalform von $f$
ist. Dies ist gleich dem gewöhnlichen Skalarprodukt
\begin{align*}
  \sprod{\Gamma^k_{ij}f_k, f_l} &= \sprod{f_{ij}^\top, f_l} \\
  &= \sprod{f_{ij}, f_{l}}
\end{align*}, wobei $f_i(p) := \frac{\partial f}{\partial x^i}(p) = \d f(p)\cdot
\frac{\partial}{\partial x^u}\vert_p \in \R^n$ gilt.

Nun definieren wir
\begin{align*}
  g_{jl,i} &:= \frac{\partial}{\partial x^i}\sprod[p]{\frac{\partial}{\partial
  x^j},\frac{\partial}{\partial x^l}} = \frac{\partial}{\partial x^i}\sprod{f_j,
  f_l} \\
  &= \sprod{f_{ij}, f_l} + \sprod{f_j, f_{il}}
\end{align*}

Dafür gilt folgendes:
\begin{align*}
  g_{ik,j} + g_{jk,i} - g_{ij,k} &= 2\sprod{f_{ij}, f_k}
\end{align*}
was man einfach durch aufsummieren erhält.

Nun sei $k \to l$ (Indexumbenennung). Dann ist
\begin{align*}
  g_{il,j} + g_{jl,i} - g_{ij,l} = 2\sprod{f_{ij}, f_l} = 2\Gamma^k_{ij}g_{kl}
\end{align*}

Bezeichne mit $g^{ij}$ die Koeffizienten der zu $(g_{ij})$ inversen Matrix, also
$g_{ij}g^{jk} = \delta^k_i$. Damit erhalten wir nämlich
\begin{align*}
  \Gamma^m_{ij} &= \Gamma_{ij}^kg_{kl}g^{lm} \\
  &= \frac{1}{2}g^{lm}\left(g_{il,j} + g_{jl,i} - g_{ij,l}\right)
\end{align*}

Seien $X,Y$ Vektorfelder auf $M$ (oder einer offenen Teilmenge von $M$). Dann
ist $\d f\cdot Y:U\to\R^n$, $p\mapsto\d f(p)\cdot Y(p)$ ein \emph{tangentiales
Vektorfeld} längs $f$, $\d f(p)\cdot Y(p)\in T_pf$. Die Ableitung von $\d f
\cdot Y$ in Richtung $X$ ist eine glatte Abbildung $U\to \R^n$ aber im
Allgemeinen nicht mehr tangential längs $f$.

Wir zerlegen nun
\begin{align*}
  X(\d f\cdot Y) = \left( X(\d f\cdot Y) \right)^\top + \left( X(\d f \cdot Y)
  \right)^\perp
\end{align*}
also $\left( X_p(\d f\cdot Y) \right)^\top, \left( X_p(\d f\cdot Y)
\right)^\perp\in N_pf$.

Über $U$ wie oben schreibe $X = \xi^i\frac{\partial}{\partial x^i}$ und $Y =
\eta^i\frac{\partial}{\partial x^i}$.

Dann ist 
\begin{align*}
  \d f\cdot Y &= \d f\left( \eta^i\frac{\partial}{\partial x^i} \right) \\
  &= \eta^i\d f\cdot\frac{\partial}{\partial x^i} = \eta^if_i
\end{align*}
und
\begin{align*}
  X(\d f\cdot Y) &= X(\eta^if_i) = X(\eta^i)f_i + \eta^iX(f_i) \\
  &= X(\eta^i)f_i + \xi^i\eta^j\frac{\partial f_j}{\partial x^i} \\
  &= X(\eta^i)f_i + \xi^i\eta^jf_{ij} \\
  &= \xi^i\frac{\partial\eta^j}{\partial x^i}f_j + \xi^i\eta^jf_{ij}
\end{align*}

Nun können wir $f_{ij}$ in den Christoffelsymbolen ausdrücken. Über $U$ ist
damit
\begin{align*}
  \left(X(\d f\cdot Y)\right)^\top &= X(\eta^i)f_i + \xi^i\eta^jf_{ij}^\top \\
  &= \left( X(\eta^k) + \xi^i\eta^j\Gamma^k_{ij} \right)f_k
\end{align*}
Der zweite Term in der Klammer ist dabei ein Korrekturterm.

\begin{dfn}[kovariante Ableitung]
  Die \emph{kovariante Ableitung} von $Y$ in Richtung $X$ ist definiert durch
  \begin{align*}
    \d f(\nabla_XY) = \left( X(\d f\cdot Y) \right)^\top
  \end{align*}
  Mit Koordinaten $x$ über $U$ wie oben ist
  \begin{align*}
    \nabla_XY := \left( X(\eta^k) + \xi^i\eta^j\Gamma^k_{ij}
    \right)\frac{\partial}{\partial x^2}
  \end{align*}
  wobei $X = \xi^i\frac{\partial}{\partial x^i}$, $Y =
  \eta^i\frac{\partial}{\partial x^i}$ sind. Damit ordnet der
  \emph{Zusammenhang} $\nabla$ glatten Vektorfeldern $X, Y$ das glatte
  Vektorfeld $\nabla_XY$ zu.
\end{dfn}

\begin{enumerate}
  \item Seien jetzt $X,Y,Z$ glatte Vektorfelder auf $M$.
    Dann ist $\sprod{Y,Z}:M\to\R, p\mapsto\sprod[p]{Y(p), Z(p)}$ eine glatte
    Funktion und (nach der Definition der ersten Fundamentalform
    \begin{align*}
      X\sprod{Y,Z} &= X\sprod{\d f\cdot Y, \d f\cdot Z} = \sprod{\left( X(\d
      f\cdot Y) \right)^\top, \d f\cdot Z} + \sprod{\d f\cdot Y, \left( X(\d
      f\cdot Z \right)^\top} \\
      &= \sprod{\d f\cdot\nabla_XY,\d f\cdot Z} + \sprod{\d f\cdot Y, \d
      f\cdot\nabla_XZ} \\
      &= \sprod{\nabla_XY,Z} + \sprod{Y,\nabla_XZ}
    \end{align*}, das heißt, $\nabla$ ist \emph{metrisch} (oder auch
    \emph{riemannsch}):
    \begin{align*}
      X\sprod{Y,Z} = \sprod{\nabla_XY,Z} + \sprod{Y,\nabla_XZ} &\quad&
      \text{(Produktregel)}
    \end{align*}

  \item Mit Koordinaten $x$ über $U$ und Vektorfeldern $X,Y$ wie oben gilt über
    $U$
    \begin{align*}
      \nabla_XY - \nabla_YX &= \left( X(\eta^k) + \xi^i\eta^j\Gamma^k_{ij} -
      Y(\xi^k) - \eta^j\xi^i\Gamma_{ji}^k \right)\frac{\partial}{\partial x^k}
      \\
      &= \left( X(\eta^k) - Y(\xi^k) \right)\frac{\partial}{\partial x^k} =
      \left[ X, Y \right]
    \end{align*}, das heißt $\nabla$ ist \emph{symmetrisch} (oder auch
    \emph{torsionslos}).

  \item Ferner gilt: $\nabla_XY$ ist \R-linear in $X$ und in $Y$.
  \item \ldots und falls $\phi:M\to\R$ glatt ist
    \begin{align*}
      &\nabla_{\phi X}Y &= \phi\nabla_XY
      \text{und} & \nabla_X(\phi Y) &= X(\phi)Y + \phi\nabla_XY
    \end{align*}
  \item Falls $X_1(p) = X_2(p)$ für ein $p\in M$, so ist
    \begin{align*}
      \left( \nabla_{X_1}Y \right)(p) = \left( \nabla_{X_2}Y \right)(p)
    \end{align*}
\end{enumerate}

$\nabla$ heißt auch der zur ersten Fundamentalform gehörige
\emph{Levi-Civita-Zusammenhang}. Die für den Zusammenhang im Prinzip
ausreichenden Eigenschaften sind 3 und 4, für \emph{Levi-Civita} nötig sind 1
und 2.

\subsection{Noch etwas zur geodätischen Differentialgleichung}
Seien $X,Y$ Vektorfelder auf $M$.
\begin{align*}
  X(\d f\cdot Y) &= \d f\cdot\nabla_XY + \left( X(\d f\cdot Y) \right)^\perp \\
  &= \d f\cdot\nabla_XY + S(X,Y)
\end{align*}
wobei $S$ die \emph{zweite Fundamentalform} von $f$ ist. Die zweite
Fundamentalform ist \R-linear in $X$ und in $Y$ und ferner ist
\begin{enumerate}
  \item $S(\phi X, Y) = \phi S(X,Y)$, $\phi:M\to\R$ glatt
  \item $S(X,Y) = S(Y,X)$
\end{enumerate}
\subsubsection{Beweise dazu:}
\begin{align*}
  S(\phi, X, Y) &= \phi X(\d f\cdot Y) - \d f(\nabla_{\phi X}Y) \\
  &= \phi\left( X(\d f\cdot) - \d f(\nabla_XY) \right) \\
  &= \phi S(X,Y)
\end{align*}
und 
\begin{align*}
  S(X,Y) -S(Y,X) &= \left( X(\d f\cdot Y) - Y(\d f\cdot X) - \d f(\nabla_XY) +
  \d f(\nabla_YX) \right) \\
  &= \d f\left( \left[ X,Y \right] - \nabla_XY + \nabla_YX \right) = 0
\end{align*}

\section{Untermannigfaltigkeiten in $\R^N$: Diskussion einiger Beispiele}
\subsection{Die Lie-Gruppen $SO(n)$ und $SU(n)$}
Auf dem Vektorraum $\R^{n\times n}$ bzw.\ $\C^{n\times n}$ definieren wir das
Skalarprodukt
\begin{align*}
  \sprod{X, Y} := \operatorname{tr}(X^*Y) \text{ für } X,Y\in\R^{n\times n}
  \text{ bzw.\ } \C^{n\times n}
\end{align*}
oder auch ausführlicher
\begin{align*}
  \sprod{X,Y} := \sum^n_{i,j=1} \bar{X}_{ij}Y_{ij}
\end{align*}

\subsubsection*{Zur Erinnerung}
\begin{align*}
  SO(n) &= \mdef[g\in\R^{n\times n}]{gg^*=1, \det g = 1 } \subset \R^{n\times n}
  \\
  SU(n) &= \mdef[g\in\C^{n\times n}]{gg^*=1, \det g = 1 } \subset \C^{n\times n}
\end{align*}
Beides sind Untermannigfaltigkeiten von $\R^{n\times n}$ bzw.\ $\C^{n\times n}$.

Es ist
\begin{align*}
  \d f_g(gX = g(X + X^*)g^*
\end{align*}

Wähle einen Weg $\gamma:\R \to \R^{n\times n}$ mit $\gamma(0) = g$, $\dot
\gamma(0) = gX$, z.~B.\ $\gamma(t) = g\exp(tX)$.

Dann ist
\begin{align*}
  \frac{\d}{\d t}\vert_{t=0}f(\gamma(t)) &= \frac{\d}{\d t}\vert_{t=0}\left(
  g\exp(tX)\exp(tX^*)g^* - 1 \right) \\
  &= gXg^* + gX^*g^* \\
  &= g(X + X^*)g^*
\end{align*}

Aufgrund der Definition des Skalarprodukts ist die erste Fundamentalform auf $G
= SO(n)$ bzw.\ $G = SU(n)$ gegeben durch
\begin{align*}
  I(gX, gY) &= \tr\left( (gX)^*gY \right) = \tr\left( X^*g^*gY \right) \\
            &= \tr(X^*Y) & \text{Killing-Form}
\end{align*}
für $X,Y \in T_1G$.

\begin{lem}
  Sei $g_0\in G$, $X\in T_1G$. Die (eindeutig bestimmte) Geodäte $g:\R\to G$ mit
  $g(0) = g_0$ und $\dot g(0) = gX$ ist gegeben durch
  \begin{align*}
    g(t) = g_0\exp(tX), \quad t\in\R
  \end{align*}

  \begin{proof}
    Berechne
    \begin{align*}
      \dot g(t) &= g_0\exp(tX) X = g(t)\cdot X, \\
      \ddot g(t) &= g(t)\cdot X^2
    \end{align*}
    Es folgt für beliebiges $t\in\R$ und $g(t)Y \in T_{g(t)}G$:
    \begin{align*}
      \sprod{g(t)Y, g(t)X^2} = \tr(Y^*X^2) = 0
    \end{align*}
    Damit folgt aus dem Kriterium von Bernoulli, dass $g$ eine Geodäte ist.
  \end{proof}
\end{lem}

\newcommand{\zwei}{\ensuremath{\mathrm{I\!I}}}
\begin{prp}
  Sei $M\subset\R^N$ eine Untermannigfaltigkeit und $c:I\to M$ eine Geodäte.
  Dann gilt
  \begin{align*}
    \zwei(\dot c(t), \dot c(t)) = \ddot c(t) \text{ für } t\in I
  \end{align*}
  \begin{proof}
    \begin{align*}
      \ddot c(t) &= \ddot c(t)^\top + \ddot c(t)^\perp \\
      &= \nabla_{\dot c}\dot c(t) + \zwei(\dot c(t), \dot c(t))
      \Rightarrow \ddot c(t) &= \zwei(\dot c(t), \dot c(t))
    \end{align*}
    Der erste Term der Summe (die tangentiale Komponente) wird $0$, da $c$ eine
    Geodäte ist.

    Weil \zwei symmetrisch ist gilt:
    \begin{align*}
      \zwei(X,Y) &= \zwei(X+Y-Y, X+Y-X) \\
                 &= \zwei(X+Y,X+Y) + \zwei(X+Y, -X) + \zwei(-Y,X+Y) +
                 \zwei(-Y,-X)
    \end{align*}
  \end{proof}
\end{prp}

\subsubsection*{Zurück zur Lie-Gruppe}
Sei $g(t) = g_0\exp(tX)$, $t\in\R$ eine Geodätische in $G$ wie oben. Dann gilt
nach der Proposition $\zwei(\dot g(0), \dot g(0)) = \zwei(g_0X,g_0X) = g_0X^2)$,
also (nach Polarisierung) $\zwei(g_0X,g_0Y) = \frac{1}{2}g_0(XY+YX)$ für
$X,Y\in T_1G$.

\subsection{Graßmann-Mannigfaltigkeiten}
Sei $(V,\sprod{\cdot,\cdot})$ ein endlichdimensionaler \R- oder \C-Vektorraum
mit Skalarprodukt. Es bezeichne $G_k(V)$, $k = 0,\ldots,\dim V$,  die Menge der
$k$-dimensionalen Unterräume von $V$.

Auf $G_k(V)$ definieren wir einen differenzierbaren Atlas wie folgt:

Sei $X\in G_k(V)$, $Y\subset V$ ein direktes Komplement von $X$ (z.~B.\
$Y=X^\perp$), also $V = X \oplus Y$. Ferner sei
\begin{align*}
  U_Y := \mdef[Z\in G_k(V)]{Z \cap Y = \mdef{0}}
\end{align*}
sowie
\begin{align*}
  f_Y : U_y \to \operatorname{Hom}(X, Y), \quad Z\mapsto \operatorname{pr}_Y(Z)
  \in \operatorname{End}(V)
\end{align*}
Dabei ist $\operatorname{pr}_Y(Z)$ die Projektion auf $Z$ längs $Y$.
% Matrix bezüglich der Basis x1, \ldots, xn, y1, \ldots, ym

Nach Konstruktion sind $G_k(V)$ von Mengen der Form $U_Y =
f_Y^{-1}(\mathbb{K}^{(n-k)\times k})$. Diese bilden zusammen mit den $f_Y$
einen differenzierbaren Atlas (wobei noch zu zeigen wäre, dass $f_{Y_2}\circ
f_{Y_1}^{-1}$ glatt ist).

$G_k(V)$ mit der so definierten differenzierbaren Struktur heißt
\emph{Graßmann-Mannigfaltigkeit} der $k$-dimensionalen Unterräume in $V$.

\begin{bem}
  Sei $V=\R^n$. Die Gruppe $G = O(n)$ wirkt auf $G_k(V)$ durch
  \begin{align*}
    gX = \mdef[gv]{v\in X}
  \end{align*}.
  Die Wirkung ist transitiv (\emph{Übungsaufgabe}). Der Stabilisator von
  $X=\operatorname{span}(e_1,\ldots,e_k)$ ist die Gruppe $O(k)\times O(n-k)
  \subset O(N)$.
\end{bem}<++>

\end{document}

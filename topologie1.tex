\documentclass{skript}

\begin{document}

\title{Topologie I}
\author{}

\maketitle

\tableofcontents

\section{Grundlagen}
\begin{dfn}
    Eine Abbildung $d: X \times X \to \R$ heißt Metrik, wenn
    $\forall x,y,z \in X$
    \begin{itemize}
    \item $d(x, y) = 0 \quad \Equiv \quad x = y$
    \item $d(x, y) = d(y, x)$
    \item $d(x, z) \leq d(x, y) + d(y, z)$
    \end{itemize}
    Ein Paar $(X, d)$ aus einer Menge $X$ und einer Metrik $d$ auf dieser
    Menge heißt \emph{metrischer Raum}.    
\end{dfn}

\begin{bsp}[euklidische diskrete Metrik]
\end{bsp}

\begin{bsp}[euklidische Metrik]
\end{bsp}

\begin{bsp}[Unterraummetrik]
\end{bsp}

\begin{bsp}
    $M \subset \R$
\end{bsp}

\begin{dfn}[Stetigkeit]
    Eine Abbildung $f: X \to Y$ auf metrischen Räumen $(X,d_X)$ und $(Y, d_Y)$
    heißt \emph{stetig im Punkt $x_0$}, falls für alle $\epsilon$ ein $\delta$
    existiert so dass gilt:
    \begin{align*}
        d_X(x_0, x) < \delta& &\Rightarrow& &d_Y(f(x_0), f(x)) < \epsilon
    \end{align*}
    Die Abbildung heißt stetig, wenn dies für alle $x_0\in X$ gilt.
\end{dfn}

\begin{dfn}[Folgenkonvergenz]
    Eine Folge $(a_n)$ mit Werten im metrischen Raum $(X, d)$ konvergiert gegen den
    Grenzwert $a$, falls
    \begin{align*}
        \forall \epsilon > 0\ \exists N,\ &\mathrm{s.d.}& \forall n > N: d(a, a_n)
        < \epsilon
    \end{align*}
\end{dfn}

\begin{stz}
    Eine Folge hat immer höchstens einen Grenzwert.
\end{stz}

\begin{stz}
    Eine Abbildung $f: X \to Y$ auf metrischen Räumen $(X, d_X)$ und $(Y, d_Y)$ ist
    genau dann stetig in $x_0$, wenn für alle in $X$ gegen $x_0$ konvergenten Folgen
    $x_n$ gilt: \[\lim_{n\to \infty} f(x_n) = f(x_0)\]
\end{stz}

\begin{axm}[Auswahlaxiom]
    Sei $B$ eine Menge. Dann $\exists c: B \to \bigcup_{x\in B} X$ mit $c(x)\in X$.   
\end{axm}

\begin{dfn}
    Sei $X$ eine Menge, $d_1$, $d_2$ Metriken auf X. $d_1$ und $d_2$ heißen
    \emph{äquivalent} falls
    \[id: (X, d_1) \to (X, d_2) \quad \mathrm{und}\quad id: (X, d_2) \to (X, d_2)\]
    stetig sind.

\begin{bem}
    $d_1$ und $d_2$ seien äquivalente Metriken auf $X$, $(Y, d')$ metrischer Raum
    \[\{f:(X, d_1) \to (Y, d') | f \mathrm{stetig} \} = \mathrm{bla}\]
\end{bem}
\end{dfn}

\begin{stz}
    Sei $X$ eine Menge, $d_1$ und $d_2$ Metriken. Existieren $c_1, c_2: X\to
    \R^+$ so dass
    \[c_1(x)d_1(x, y) \leq d_2(x,y) \leq c_2(x)d_1(x,y) \forall x,y \in X,\]
    dann sind $d_1$ und $d_2$ äquivalent.
\end{stz}

\begin{dfn}
    Eine \emph{Norm} auf dem \menge{K}-Vektorraum $V$ ist eine Abbildung
    $\norm{\cdot}: V \to \R^+_0$ mit den Eigenschaften
    \begin{itemize}
        \item $\norm{x} = 0 \quad \Equiv \quad x = 0$
        \item $\norm{\lambda x} = |\lambda| \norm{x} \
                    \quad \forall x \in V, \lambda \in \menge{K}$
        \item $\|x + z\| \leq \|x + y\| + \|y + z\| \quad \forall x,y,z \in V$
    \end{itemize}
    Das Paar $(X, \norm{\cdot})$ heißt \emph{normierter Raum}.
\end{dfn}

\begin{stz}
    Sei $(V, \norm{\cdot}$ ein normierter Raum, dann ist $d(u,v) := \norm{u - v}$
    eine Metrik auf $V$.
\end{stz}

\begin{dfn}
    Sei $A$ eine Menge, $\R^A$ der \R-Vektorraum aller Funktionen $\R \to A$,
    $B(A) = \mdef[f\in\R^A]{\exists k\in\N\!: \norm{f(a)} < k \ \forall a \in A}$
    der Unterraum der beschränkten Funktionen und $\supnorm{\cdot}: B(A) \to \R$
    mit $\supnorm{f} := \sup \mdef[\norm{f(a)}]{a\in A}$. Dann ist
    $\supnorm{\cdot}$ eine Norm auf $B(A)$.
\end{dfn}

\begin{bsp}
    $A = \mdef{1,\ldots,n}\subset \N$, dann gilt $B(A) = \R^n$ und wir haben
    die Metrik $d_\infty: \R^n \times \R^n \to \R$, die von $\supnorm{\cdot}$
    erzeugt wird. $d_\infty$ und die euklidische Metrik sind äquivalent.
    Es gilt:
    \[d_\infty(x,y) \le d(x,y) \le \sqrt{n}\ d_\infty(x,y) \quad \forall x,y\in\R^n\]
\end{bsp}

\newtheorem{wbsp}[dfn]{weitere Beispiele}
\begin{wbsp}
    \begin{itemize}
        \item $\ell^\infty = B(\N)$ mit $\supnorm{\cdot}$
        \item $\ell^p \subset B(\N)$ mit
                $\norm[p]{f} = \sqrt[p]{\sum^\infty_{i=1} \norm{f(i)}^p}$
        \item $\R^\N = \mdef[f \in B(\N)]{bla}$
    \end{itemize}
\end{wbsp}

\begin{dfn}
    (gleichmäßige und punktweise Konvergenz)
\end{dfn}

\begin{stz}
    Konvergiert $(f_n)_{n\in\N} \to f$ gleichmäßig, dann ist falls alle $f_n$
    stetig sind auch $f$ stetig.
\end{stz}

\begin{dfn}
    Kugeln
\end{dfn}

\begin{dfn}
    Offen
\end{dfn}

\begin{lem}
    Sei $(X, d)$ ein metrischer Raum. Dann ist $B_\epsilon(x)$ offen
    $\forall \epsilon > 0$.
\end{lem}

\begin{lem}
    X und leere Menge sind offen.
\end{lem}

\begin{stz}
    Sei $f: (X, d) \to (Y, d')$ eine Abbildung zwischen metrischen Räumen. Dann sind
    äquivalent:
   \begin{itemize}
        \item $f$ ist stetig in $x_0$
        \item Zu jeder Umgebung $U$ von $f(x_0)$ gibt es eine Umgebung $W$ von $x_0$
              mit $f(W)\subset V$
    \end{itemize}
\end{stz}

\begin{stz}
    Sei $f: (X, d) \to (Y, d')$ eine Abbildung zwischen metrischen Räumen. Dann sind
    äquivalent:
    \begin{itemize}
        \item $f$ ist stetig
        \item $\forall U \subset Y$ offen ist auch $f^{-1}(U)$ offen
    \end{itemize}
\end{stz}

\begin{krl}
    $(X,d) \overset{f}{\to} (Y,d') \overset{g}{\to} (Z, d'')$ mit $f,g$ stetig, dann
    ist auch $g \circ f$ stetig.
\end{krl}

\begin{bsp}
    Die Abbildung $\chi_{(-\infty,0]}$ ist nicht stetig in 0. Es gibt keine Umgebung
    von 0, so dass $f(W) \subset B_1(f(0))$.
\end{bsp}

\begin{bem}
    Die Menge der offenen Teilmengen von Räumen genügt, um Stetigkeit zu
    charakterisieren.
\end{bem}

\begin{stz}
    Seien $d_1$ und $d_2$ Metriken auf $X$. Dann sind äquivalent:
    \begin{itemize}
        \item $d_1$ und $d_2$ sind äquivalente Metriken
        \item Die Menge $U \subset X$ ist genau dann in $(X, d_1)$ offen, wenn sie
              auch in $(X, d_2)$ offen ist.
    \end{itemize}
\end{stz}

\begin{stz}
    $(X,d)$ sei metrischer Raum und $T = \mdef[U\subset X]{U\ \mathrm{offen}}$.
    Dann gelten:
    \begin{enumerate}
        \item $\emptyset, X \in T$
        \item $I \subset T$, dann gilt $\bigcup_{U\in I} U \in T$
        \item $\mdef{U_1, \ldots, U_n} \subset T$, dann gilt
              $\bigcap_{i=1}^n \in T$
    \end{enumerate}
\end{stz}

\begin{stz}
    Sei $(X, d)$ metrischer Raum.
    \begin{enumerate}
        \item $x,y \in X$, $x \ne y$, dann existieren Umgebungen $U$ von $x$ und $V$
              von $y$ so dass $U \cap V = \emptyset$.
        \item $x \in X$, dann existiert eine Folge $(U_n)_{n\in \N}$ von Umgebungen
            von x so dass für jede Umgebung $V$ von $x$ ein $N\in\N$ existiert mit
            $U_n \subset V$ für $n > N$.
    \end{enumerate}
\end{stz}

\begin{stz}
    $(X, d)$ metrischer Raum, $U \subset X$. Dann sind äquivalent:
    \begin{itemize}
        \item $U$ ist offen
        \item $U$ ist eine Vereinigung offener Kugeln
    \end{itemize}
\begin{bem}
    Sei $(X, d)$ diskreter Raum, dann bla
\end{bem}
\end{stz}

\section{Topologie}

Notation: X eine Menge, $\mathcal{P}(X)$ die Potenzmenge von X.

\begin{dfn}
    Eine \emph{Topologie} $\tau$ auf einer Menge X ist eine Teilmenge
    $\tau\subset\mathcal{P}(X)$ mit
    \begin{enumerate}
        \item $\emptyset, X \in \tau$
        \item Ist $I\subset \tau$, dann $\bigcup_{U\in I} U \in\tau$
        \item $\mdef{U_1, \ldots, U_n} \subset \tau$, dann
            $\bigcup_{i=1}^n U_i \in \tau$
    \end{enumerate}
    Das Paar $(X, \tau)$ ist dann ein \emph{topologischer Raum}, $U\in\tau$ heißt
    eine \emph{offene Teilmenge von $X$}.
\end{dfn}

\begin{dfn}
    Sei $X$ eine Menge, $\tau_1$ und $\tau_2$ Topologien auf $X$.
    \begin{itemize}
        \item $\tau_1$ heißt \emph{feiner} als $\tau_2$, falls $\tau_2\subset\tau_1$
        \item $\tau_1$ heißt \emph{strikt feiner} als $\tau_2$, falls
            $\tau_2\subsetneq\tau_1$
        \item $\tau_1$ und $\tau_2$ heißen \emph{vergleichbar}, falls
            $\tau_1\subset\tau_2$ oder $\tau_2\subset\tau_1$
    \end{itemize}
\end{dfn}

\begin{bsps}
    \begin{enumerate}
        \item $(X,d)$ metrischer Raum und sei $\tau_d$ die Menge der offenen
            Teilmengen von $X$. Dann ist $\tau_d$ die \emph{metrische Topologie}
            auf $(X,d)$.
        \item $\mathcal{P}(X)$ ist eine Topologie auf $X$ (diskrete Topologie). Es
            ist die feinste aller Topologien.
        \item $\tau = \mdef{\emptyset, X}$ definiert eine Topologie auf X. Sie
            heißt die \emph{grobe Topologie}.
        \item $X=\mdef{1,2,3}$. Dann sind $\tau_1=\mdef{\emptyset,X,\mdef{1}}$ und
            $\tau_2=\mdef{\emptyset, X, \mdef{2}, \mdef{2,3}}$ Topologien auf $X$,
            die nicht vergleichbar sind. Dagegen ist
            $\mdef{\emptyset, X, \mdef{1}, \mdef{2}, \mdef{2,3}}$ keine Topologie.
        \item Sei $X$ eine Menge und
            $\tau = \mdef{\emptyset}\cup\mdef[U\subset X]{X\setminus U \
            \mathrm{ist endlich}}$.  Das definiert eine Topologie auf $X$, die
            \emph{Endliche-Komplemente-Topologie}
    \end{enumerate}
\end{bsps}

Im Fall von metrischen Räumen hat man die Topologie mit Hilfe der offenen Kugeln
definiert, ähnlich\begin{dfn}
    Sei $X$ eine Menge. Eine \emph{Basis} $B$ einer Topologie auf $X$ ist eine
    Teilmenge $B\subset\potmenge{X}$ mit
    \begin{itemize}
        \item $\forall x\in X\ \exists U\in B$ mit $x\in U$
        \item Falls $U_1,U_2\in B$ und $x\in U_1 \cap U_2$, dann existiert ein
            $U_3 \in B$ mit $x\in U_3 \subset U_1\cap U_2$
    \end{itemize}
\end{dfn}

\begin{stz}
    Sei $X$ eine Menge, $B\subset\potmenge{X}$ eine Basis. Sei
    $\tau\subset\potmenge{X}$ mit $U\in\tau$ genau dann, wenn es für jedes $x\in U$
    ein $V\in B$ gibt, mit $x\in V\subset U$. Dann ist $\tau$ eine Topologie auf $X$.
    Sie heißt die \emph{von $B$ erzeugte Topologie}.

    \begin{proof}[Beweis]
        \begin{itemize}
            \item $\emptyset\in\tau$ klar.
            \item $X\in\tau$ klar (2.4).
            \item bla
        \end{itemize}
    \end{proof}

\end{stz}

\begin{lem}
    Sei $X$ eine Menge, $B\subset\potmenge{X}$ eine Basis und $\tau$ die von $B$
    erzeugte Topologie. Dann gilt: \[\tau = \mdef[\cup_{U\in I}U]{I\subset B}\]

    \begin{proof}[Beweis]
        Analog zu Lemma 1.35 (ref).
    \end{proof}
\end{lem}

\begin{lem}
    Sei $(X, \tau)$ ein topologischer Raum. Sei $B\subset \tau$ mit der Eigenschaft:
    F"ur jedes $U\in \tau$ und jedes $x\in U$ existiert $V\in B$ mit $x\in V\subset U$.
    Dann ist $B$ eine Basis, die $\tau$ erzeugt.

    \begin{proof}
        \begin{enumerate}
            \item $X$ ist offen, also $\forall x\in X \exists V\in B$ mit $x\in V$.
                Damit gilt 2.4(1) (ref).
            \item Sei $x\in V_1\cap V_2$ mit $V_1, V_2 \in B \Rightarrow V_1, V_2 \in \tau$.
                also $x\in V_1\cap V_2$ offen, also $\exists V_3$ mit
                $x\in V_3\subset V_1\cap V_2$ (2.4(2) ref).
            \item Sei $\tau'$ die Topologie, die von $B$ erzeugt wird. Wir m"ochten zeigen:
                $\tau' = \tau$.\\
                Sei $U\in\tau'$, dann ist $U$ Vereinigung von Elementen in $B$
                (Lemma 2.6 ref),
                aber alle solche sind offen in $\tau$, also gilt $U\in\tau$.\\
                Umgekehrt: Sei $U\in\tau$, $x\in U$, dann existiert $V\in B$ mit
                $x\in V\subset U$, also ist $U$ offen bez"uglich $\tau'$.
        \end{enumerate}
    \end{proof}
\end{lem}

\begin{lem}
    Seien $B_1$, $B_2$ Basen von Topologien $\tau_1$, $\tau_2$ auf $X$. Dann ist $\tau_1$
    genau dann feiner als $\tau_2$, wenn f"ur jedes $U\in B_2$ mit $x\in U$ ein $V\in B_1$
    existiert mit $x\in V\subset U$.
    \begin{proof}
        \subsubsection*{Hinrichtung}
        Sei $x\in X$, $U\in B_2$ mit $x\in U$, da $\tau_2 \subset \tau_1$ ist $U$ offen in
        $\tau_2$ (?!). Damit existiert per Definition (von was?) ein $V\in B_1$ mit
        $x\in V\subset U$.
        \subsubsection*{R"uckrichtung}
        Sei $W\in\tau_2$ und sei $x\in W$. Dann existieren $V\in B_2$ und $x\in V\subset W$.
        Damit existiert nach Annahme ein $U\in B_1$ mit $x\in U\subset V$. Also
        $X\in W \subset U$ und $W\in \tau_1$.
    \end{proof}
    \begin{bsp}
        $(X,d)$ metrischer Raum. Dann ist $\mdef[B_\eps(x)]{x\in X,\ \eps > 0}$ eine Basis
        f"ur die metrische Topologie.
    \end{bsp}
    \begin{bsp}
        $X = \R$ und $B_l = \mdef[[a,b)]{a,b\in \R,\ a < b}$. Dann ist $B_l$ eine Basis, die
        entsprechende Topologie hei\ss{}t die $blabla$ Topologie $\tau_l$.
        \begin{bem}
            $T_l$ ist strikt feiner als die euklidische Topologie.
        \end{bem}
    \end{bsp}
\end{lem}

\begin{dfn}
    Eine Teilmenge $S\subset \potmenge{X}$ ist eine \emph{Subbasis} einer
    Topologie auf $X$, falls \[ \forall x\in X \exists U\in S \ \mathrm{mit}\
    x\in U \].
    \begin{stz}
        Die Teilmenge $B \subset \potmenge{X}$, $B := \mdef[U_1 \cap \ldots
        \cap U_n]{n\in \N,\ U_i \in S}$ ist eine Basis. Die erzeugte Topologie
        hei\ss{}t von der Subbasis $S$ erzeugt.
        \begin{proof}
            Klar
        \end{proof}
    \end{stz}
    \begin{bsp}[$\R^n$ mit der euklidischen Topologie]
        Sei f"ur $i\in \underbar{n} = {1, \ldots, n}$ und $\alpha\in\R$
        \[S^+(i, \alpha)=\mdef[x\in\R^n]{x_i>\alpha}\]
        \[S^-(i, \alpha)=\mdef[x\in\R^n]{x_i<\alpha}\]
        Dann ist $S=\mdef[S^+(i,\alpha)]{i\in\underbar n,\ \alpha\in\R}
        \cup \mdef[S^-(i,\alpha)]{i\in\underbar n,\ \alpha\in\R}$
        eine Subbasis der euklidischen Topologie.
    \end{bsp}
    \begin{bsp}[Punktweise Konvergenz]
        Sei $(X,\tau)$ ein topologischer Raum und $\Omega$ eine Menge. Man
        definiert eine Topologie auf $X^\Omega = \mdef{f:\Omega\to X}$ durch die
        Angabe einer Subbasis:
        \[S=\mdef[s(\omega, U)]{\omega\in\Omega,\ U\in\tau} \mathrm{ mit }
        s(\omega, U)=\mdef[f:\Omega\to X]{f(\omega)\in U}\]
        Es gilt $s(\omega, X)=X^\Omega$ also ist $S$ eine Subbasis, zum Beispiel
        f"ur $X^\Omega = \R^\R$. Die so definierte Topologie hei\ss{}t die
        \emph{Topologie der punktweisen Konvergenz}.
    \end{bsp}
\end{dfn}

\begin{dfn}
    Ein topologischer Raum hei\ss{}t \emph{metrisierbar}, wenn es auf $X$ eine
    Metrik $d$ gibt, mit $\tau=\tau_d$.
    \begin{bsp}
        \begin{enumerate}
            \item Die diskrete Topologie ist metrisierbar durch die diskrete Metrik
            \item Hat $X$ mehr als ein Element, so ist die grobe Topologie nicht
                metrisierbar.
            \item $\tau_l$ auf $\R$ und punktweise Konvergenz auf $\R^{(\R^n)}$ ist
                nicht metrisierbar.
        \end{enumerate}
    \end{bsp}
\end{dfn}

\begin{dfn}
    Seien $(X,\tau)$ und $(Y, \tau')$ topologische R"aume. Auf dem Produkt
    $X\times Y$ definiert man die \emph{Produkttopologie} mit Hilfe einer
    Subbasis
    \[S = \mdef[U\times V]{U\in\tau} \cup \mdef[X\times V]{V\in\tau'}\]
    mit entsprechender Basis
    \[B = \mdef[U\times V \subset X\times Y]{U\in\tau,\ V\in\tau'}\]

    \begin{bem}
        \begin{enumerate}
            \item $B$ selbst ist keine Topologie auf $X\times Y$, z.B. $(0,\ 1)^2
                \cup (1,\ 2)^2 \subset \R^2$.
            \item Seien $B_1, B_2$ (bzw. $S_1, S_2$) Basen (bzw. Subbasen) f"ur
                $\tau,\tau'$, dann ist
                \[B = \mdef[U\times V]{U\in B_1,\ V\in B_2}\]
                eine Basis (bzw. entsprechend mit $S_{1,2}$ Subbasis) von $X\times Y$.
        \end{enumerate}
    \end{bem}
\end{dfn}

\begin{dfn}
    Sei $(X,\tau)$ ein topologischer Raum, $X\subset Y$ eine Teilmenge, dann ist
    $\tau_Y = \mdef[U\cap Y]{U\in\tau}$ eine Topologie auf $Y$, die
    \emph{Teilraumtopologie}.

    \begin{bem}
        \begin{enumerate}
            \item Analog zu (bla, ref 2.17) f"ur Basis und Subbasis
            \item Sei $(Y,\tau_y) \subset (X,\tau)$ Teilraum. Dann ist die Aussage
                "`$U\subset X$ ist offen"' nicht ganz klar, meist bzgl. $X$. Aber
                im Allgemeinen ist $U\in\tau$ und $U\in\tau_Y$ nicht "aquivalent.
                (Beispiel)
            \item Ist $(X,d)$ metrisch, $Y\subset X$ mit induzierter Metrik $d_Y$,
                dann ist $\tau_{d_Y}$ die Unterraumtopologie.
        \end{enumerate}
    \end{bem}
\end{dfn}

\begin{bsp}
    $l^2$ bla.
\end{bsp}

\begin{dfn}
    Sei $(X. \tau)$ topologischer Raum. $A\subset X$ hei\ss{}t abgeschlossen,
    falls $(A\setminus)\in\tau$.
\end{dfn}

\begin{bsp}
    Abgeschlossene Kugel.
\end{bsp}

\begin{stz}[Dualsatz]
    Sei $(X,\tau)$ topologischer Raum.
    \begin{itemize}
        \item $X, \emptyset$ sind abgeschlossen
        \item $\mdef{A_\alpha}_{\alpha\in I}$ mit $A_\alpha$ abgeschlossen, dann ist
            $\bigcap_{\alpha\in I} A_\alpha$ abgeschlossen
        \item $A_1,\ldots, A_n$ abgeschlossen, dann ist auch $\bigcup_{i=1}^n A_i$
            abgeschlossen.
    \end{itemize}
    \begin{proof}
        De Morgan.
    \end{proof}
    \begin{bem}
        Man k"onnte topologische R"aume auch mit $\mdef[A\subset X]{A\mathrm{
        abgeschlossen}}$ mit den Punkten des Satzes als Axiomen definieren.
    \end{bem}
\end{stz}

\begin{lem}
    \begin{enumerate}
        \item Sei $(X,\tau)$ topologischer Raum, $(Y,\tau_Y)$ Teilraum, dann ist
            $A\subset Y$ abgeschlossen bezüglich $\tau_Y$, wenn es $C\subset X$
            abgeschlossen gibt mit $A=C\cup Y$.
        \item Ist $Y$ abgeschlossen in $X$, so ist $A\subset Y$ genau dann abgeschlossen
            in $Y$, wenn $A$ in $X$ abgeschlossen ist.
    \end{enumerate}
\end{lem}

\begin{dfn}
    Sei $(X,\tau)$ topologischer Raum, $A\subset X$
    \begin{itemize}
        \item Der Abschluss von $A$ in $X$ ist die Teilung
            \[\overline{A} = \bigcap\mdef[B\subset X]{B \mathrm{ abgeschlossen und }
            A\subset B}\]
        \item Dass Innere von $A$ in $X$ ist die Teilung
            \[A^\circ = \bigcup\mdef[O\subset X]{O\mathrm{ offen, } O\subset A}\]
    \end{itemize}
    \begin{bem}
        \begin{itemize}
            \item $\overline{A}$ abgeschlossen, $A$ abgeschlossen $\Equiv
                \overline{A} = A$
            \item $A^\circ$ offen, $A$ offen $\Equiv A^\circ = A$
        \end{itemize}
    \end{bem}
\end{dfn}

\begin{stz}
    Sei $(X,\tau)$ ein topologischer Raum, $(Y,\tau_Y)$ ein Teilraum und $A\subset Y$
    eine Teilmenge. Dann gilt:
    \begin{enumerate}
        \item $\overline{A}^Y = \overline{A}^X \cap Y$
        \item $A^{\circ^X} \cap Y \subset A^{\circ^Y}$ und $\supset$ genau dann, wenn
            $Y$ offen.
    \end{enumerate}
    \begin{proof}
        Klar.
    \end{proof}
\end{stz}

\begin{stz}
    Den Abschluss $A\subset(X,\tau)$ kann man wie folgt charakterisieren:
    
    \[x\in \overline{A} \Equiv \forall U\in \tau\mathrm{ (oder einer Basis von
    \tau) mit } x\in U \mathrm{ gilt } U\cap A \neq \emptyset\]
    \begin{proof}
        \begin{align*}
            & x\in\overline{A} \Equiv x\notin
            \bigcap \mdef[B\subset X]{B \mathrm{ abgeschlossen, } A\subset B} \\
            \Equiv & \exists B \mathrm{ abgeschlossen, } A\subset B,\ x\in B \\
            \Equiv & \exists U \mathrm{ offen (}U = X\setminus B\mathrm{) mit } x\in U
            \mathrm{ und } U\cap A = \emptyset
        \end{align*}
    \end{proof}
\end{stz}

\begin{dfn}
    Sei $(X,\tau)$ topologischer Raum. $V\subset X$ ist eine Umgebung von $x\in X$,
    falls $x\in V^\circ$ (muss nicht offen sein).

    \begin{bsp}
        \begin{enumerate}
            \item $(V,\norm{\cdot})$ normierter Raum, dann gilt:
                \[\overline{B}_\eps(x) \mathrm{ ist Umgebung von } x\]
            \item $\Q\in \R$ eukl., dann gilt: $\overline{\Q}=\R$
            \item $X$ mit der groben Topologie, $A\subset X$
                \[\Rightarrow A^\circ = \emptyset,\ \overline{A} = X\]
            \item $A=\mdef[\frac{1}{n}]{n = 1, \ldots}\subset\R$ eukl., dann gilt
                $\overline{A} = \mdef{0} \cup A$
        \end{enumerate}
    \end{bsp}
\end{dfn}

\begin{dfn}
    $(X,\tau)$ topologischer Raum, $A\subset X$. $x$ ist ein \emph{Berührungspunkt}
    von $A$, falls für jede offene Umgebung $V$ von $x$ gilt:
    \[V\setminus \mdef{x}\cap A\neq \emptyset\]
    \begin{bsp}
        \begin{enumerate}
            \item $\overline{B_\eps(x)}$ ist die Menge der Berührungspunkte von
                $B_\eps(x)$ im normierten Fall
            \item Jedes $x\in\R$ ist Berührpunkt von $\Q$
            \item $X$ grob, dann ist jeder Punkt $x\in X$ ein Berührpunkt von
                $\mdef{y}\subset X,\ x\neq y$
            \item $0$ ist der einzige Berührungspunkt von
                $A=\mdef[\frac{1}{n}]{n=1,\ldots}$
        \end{enumerate}
    \end{bsp}
\end{dfn}

\begin{stz}
    $(X,\tau)$ topologischer Raum, $A\subset X$. Sei $A' = \mdef[x\in X]
    {x\text{ berührt } A}$. Dann gilt $\overline{A} = A\cup A'$.

    \begin{krl}
        $A\subset (X,\tau)$ ist genau dann abgeschlossen, wenn es alle seine
        Berührungspunkte enthält.
    \end{krl}
\end{stz}

\begin{dfn}
    $(X,\tau)$, $(Y,\tau')$ topologische Räume, $f:X\to Y$ eine Abbildung.
    \begin{itemize}
        \item $f$ ist stetig am Punkt $x\in X$ wenn für jede Umgebung von $V$ von
            $f(x)$ eine Umgebung $W$ von $x$ existiert mit $f(W)\subset V$
        \item $f$ ist stetig, falls $f^{-1}(U)$ offen für alle $U\subset Y$ offen.
    \end{itemize}
    \begin{bem}
        Im metrischen Fall sind beide Punkte für alle $x\in X$ äquivalent.
    \end{bem}

    \begin{bsps}
        \begin{enumerate}
            \item $(X,\tau)\overset{id}{\to}(X,\tau')$ ist genau dann stetig, wenn
                $\tau$ feiner als $\tau'$ ist
            \item[bla] Hat $X$ die grobe Topologie, dann ist jede Abbildung
                $(Y,\tau')\to(X,\tau)$ stetig
            \item Die Inklusion $(Y,\tau_Y)\hookrightarrow (X,\tau)$ von Teilräumen
                ist stetig
            \item Konstante Abbildungen $X\overset{f}\to Y,\ f(x) = y \forall x\in X$
                sind stetig
            \item Sind $f:X\to Y$ und $g: Y\to Z$ stetig, dann ist auch $g\circ f$
                stetig
            \item Aus 3. und 5. folgt: $(X,\tau_X)\subset(X,\tau)$ und $f:Y\to Z$
                stetig, dann ist $f\Vert_X$ stetig
            \item $Y$ Teilraum von Z, ist $f:X\to Y$ stetig, so ist $f:X\to Z$ stetig
            \item $X\overset{f}\to Z$ stetig mit $f(X)\subset Y\subset Z$, so ist
                $h:X\to Y$ mit $h(x) = f(x)\ \forall x\in X$ auch stetig
            \item Gilt $X = \bigcup_{\alpha\in I}U_\alpha$ mit $U_\alpha\subset X$
                offen, dann ist $f:X\to Y$ offen, genau dann stetig, wenn
                $f\Vert_{U_\alpha}$ stetig $\forall\alpha\in I$ ist (\emph{offenes
                Klebelemma})
        \end{enumerate}
    \end{bsps}
\end{dfn}

\begin{stz}
    Für $f:(X,\tau)\to(Y,\tau')$ sind äquivalent:
    \begin{itemize}
        \item $f$ ist stetig
        \item für alle $A\subset X$ gilt $f(\overline{A}) = \overline{f(A)}$
        \item für alle $B\subset Y$ abgeschlossen ist $f^{-1}(B)$ abgeschlossen
    \end{itemize}
\end{stz}

\begin{lem}[abgeschlossenes Klebelemma]
    Sei $(X,\tau)$ ein topologischer Raum, $n\in\N$, $A_1,\ldots,A_n$ abgeschlossene
    Teilräume von $X$ mit $X=\bigcup_{i=1}^n A_i$. Sei $f:A_i\to(Y,\tau)$ stetig
    für $i = 1, \ldots, n$ so dass $f_i\Vert_{A_i\cap A_j}=f_j\Vert_{A_i\cap A_j}$
    für alle $1\leq i, j \leq n$. Dann ist die Abbildung $f : X \to Y$, $f(x) = f_i(x)$
    für $x\in A$ stetig.
    \begin{proof}
        $B\subset Y$ abgeschlossen. Es gilt: $f^{-1}(B) = \bigcup_{i=1}^n f^{-1}_i(B)$
        abgeschlossen.
    \end{proof}
    \begin{bem}
        Das Lemma gilt nicht allgemein für $\mdef{A_i}_{i\in I}$ mit $I$ unendlich,
        z.B. $f:\R\to\R$, $f(0) = 0$, $f(x) = \frac{1}{x}$ für $x\neq 0$.
        $A_0 = \mdef{0}$, $A_n = (-\infty,-\frac 1n]\cup[\frac 1n, \infty)$
        abgeschlossen, $\R=\bigcup_{i=0}^\infty A_i$, $f\Vert_{A_i}$ stetig.
    \end{bem}
\end{lem}

\begin{dfn}
    $(X,\tau)\in Top$, $\mdef{x_n}_{n\in\N}\subset X$ eine Folge in $X$ und $x\in X$.
    Man sagt $x$ ist ein \emph{Limes} von $\mdef{x_n}_{n\in\N}$, falls für jede
    Umgebung $V$ von $x$ ein $N\in\N$ existiert mit $n\geq N\Rightarrow x_n\in V$.
\end{dfn}

\begin{lem}
    $(X,\tau)$ ein topologischer Raum, $\mdef{x_n}_{n\in\N}\subset X$ und
    $x\in\lim_{n\to\infty}x_n$
    \begin{enumerate}
        \item Ist $f:X\to Y$ stetig, so gilt $f(x)\in \lim f(x_n)$
        \item Ist $A\subset X$ eine Teilmenge und $\mdef{x_n}_n \subset A$, dann gilt
            $x\in\overline{A}$
    \end{enumerate}
    \begin{proof}
        \begin{enumerate}
            \item Ist $V$ eine Umgebung von $f(x)$, dann existiert eine Umgebung
                $U$ von $x$ mit $f(U)\subset V$. $\exists N$ mit $x_n\in U\ \forall
                n > N$, also $f(x_n)\in f(U)\subset V$
            \item genauso
        \end{enumerate}
    \end{proof}
    \begin{bem}
        $(X,\tau)$ topologischer Raum, $\mdef{x_n}_n\subset X$
        \begin{itemize}
            \item $\mdef{x_n}_n$ kann verschiedene Limites haben. Ist zum Beispiel
                $\tau$ die grobe Topologie. Dann sind alle Punkte von $X$ ein $Limes$
                von $\mdef{x_n}$
            \item Die Umkehrungen des Lemmas gelten nicht.\\
                Ist $f(x) \in \lim f(x_n)$ für alle Folgen $x_n$ mit $x\in\lim x_n$,
                dann sagt man $f$ ist \emph{folgenstetig} in $x$ $\nRightarrow$
                stetig in $x$.\\
                Ebenso gilt $x\in\overline{A}$, so ist $x$ nicht notwendigerweise ein
                Limes einer Folge in $A\subset X$
        \end{itemize}
    \end{bem}
\end{lem}

\begin{stz}
    Seien $(X,\tau)$ und $(Y,\tau')$ topologische Räume, $B'$ (bzw. $S'$) eine Basis
    (bzw. Subbasis) von $\tau'$. Sei $f:(X,\tau)\to(Y,\tau')$ eine Abbildung.
    Dann sind äquivalent:
    \begin{itemize}
        \item $f$ stetig
        \item $f^{-1}(U)$ offen $\forall U\in B'$
        \item $f^{-1}(V)$ offen $\forall V\in S'$
    \end{itemize}
    \begin{proof}
        Klar (bla ref)
    \end{proof}
\end{stz}

\begin{dfn}
    $f:(X,\tau)\to(Y,\tau')$ heißt \emph{Homöomorphismus} (oder eine
    \emph{topologische Äquivalenz}), falls
    \begin{itemize}
        \item $f$ bijektiv
        \item sowohl $f$ als auch $f^{-1}$ stetig
    \end{itemize}
\end{dfn}

\begin{dfn}
    $f:(X,\tau)\to(Y,\tau')$ heißt
    \begin{itemize}
        \item \emph{offen}, falls $f(U)\in \tau'\ \forall U\in\tau$
        \item \emph{abgeschlossen}, falls $f(A)$ abgeschlossen für $A\subset X$
            abgeschlossen
    \end{itemize}
\end{dfn}

\begin{lem}
    Sei $f:(X,\tau)\to(Y,\tau')$. Dann sind äquivalent:
    \begin{enumerate}
        \item $f$ ist ein Homöomorphismus
        \item $f$ ist bijektiv, stetig und offen
        \item $f$ ist bijektiv, stetig und abgeschlossen
    \end{enumerate}
    \begin{bem}
        Eine stetige offene Abbildung muss nicht abgeschlossen sein, umgekehrt auch
        nicht.
    \end{bem}
    \begin{bsps}
        \begin{enumerate}
            \item Inverse und Verknüpfungen von Homöomorphismen sind wieder
                Homöomorphismen
            \item $a < b\in\R$, dann sind $(a,b)\subset\R$ und $\R$ homöomorph
                (eukl.):
                \begin{align}
                    f:(-1,1)\to\R,\ &f(x) = \frac{x}{1-\norm{x}} \\
                    g:(0,1)\to(a,b),\ &g(x) = (b-a)x + a
                \end{align}
        \end{enumerate}
    \end{bsps}
\end{lem}
\section{Die Quotiententopologie}

% Tolle Zeichnungen

\begin{dfn}
    Sei $X$ eine Menge. Eine \emph{Relation} auf $X$ ist eine Teilmenge
    $R\subset X\times X$, für $(x,y)\in R$ schreibt man: $x\ R\ y$ oder $x
    \sim_R y$.\\ Weitere Eigenschaften:
    \begin{enumerate}
        \item $R$ ist \emph{reflexiv}, falls $x\sim x$
        \item $R$ ist \emph{symmetrisch}, falls $x\sim y \Equiv y\sim x$
        \item $R$ ist \emph{transitiv}, falls $x\sim y,\ y\sim z \Rightarrow
            x\sim z$
        \item $R$ ist \emph{Äquivalenzrelation}, falls 1, 2 und 3 gelten.
    \end{enumerate}
\end{dfn}

\begin{dfn}
    Eine \emph{Äquivalenzklasse} von $x\in X$ ist die Teilmenge \[ [x] = [x]_R
    = \mdef[y\in X]{y \sim x} \] $X/R$ ist die Menge der Äquivalenzklassen. Die
    Abbildung $X\to X/R,\ x\mapsto[x]$ heißt \emph{kanonische Projektion}.
\end{dfn}

\begin{dfn}
    $X$ eine Menge. Eine \emph{Partition} (\emph{Zerlegung}) von $X$ ist eine Menge 
    $\mdef{A_i}_{i\in I}$ von disjunkten Teilmengen mit $X=\bigcup_{i\in I} A_i$.
\end{dfn}

\begin{lem}
    Sei $X$ eine Menge.
    \begin{enumerate}
        \item Ist $R$ eine Äquivalenzrelation auf $X$, so ist $X/R$ eine Partition
            von $X$
        \item Ist $P$ eine Partition von $X$, so existiert genau eine
            Äquivalenzrelation $F$ mit $P=X/R$
    \end{enumerate}
\end{lem}

\begin{lem}
    Ist $f:X\to Y$ eine Abbildung, so definiert
    \begin{equation}
     x \sim_f y :\Equiv f(x) = f(z)
    \end{equation}
    eine "Aquivalenzrelation auf $X$.
    \begin{proof}
     (3.4b), Urbilder bilden Partition.
    \end{proof}
\end{lem}

\begin{lem}
    Sei $f:X \to Y$ eine Abbildung und $R$ eine "Aquivalenzrelation auf $X$ so
    dass $f(x) = f(y)$ f"ur alle $x$, $y$ mit $x \sim y$. Dann existiert eine
    \emph{einzige} Abbildung $\overline{f}: X/R \to Y$, so dass das Diagramm
    kommutiert:

    \[
    \begin{diagram}
        \node{X} \arrow{e,t}{f} \arrow{s,l}{\pi}
        \node{Y}\\
        \node{X/R} \arrow{ne,r}{\overline{f}}
    \end{diagram}
    \]

    Ist $f$ surjektiv und $R$ von $f$ wie in 3.5, so ist $\overline{f}$
    bijektiv.
    \begin{proof}
        W"ahle $\overline{f}([x]) := f(x)$.
    \end{proof}
\end{lem}

\begin{lem}
    $f: X \to Y$ eine Abbildung. $\mdef{A_\alpha}_{\alpha\in I} \subset \potmenge{X}$,
    $\mdef{B_\beta}_{\beta\in J} \subset \potmenge{X}$, $D\subset Y$. Dann gilt:
    \begin{itemize}
        \item $f\left(\bigcup_\alpha A_\alpha\right) = \bigcup_\alpha f(A_\alpha)$
        \item $f\left(\bigcap_\alpha A_\alpha\right) \subset \bigcap_\alpha
            f(A_\alpha)$
        \item $f^{-1}\left(\bigcup_\beta B_\beta\right) = \bigcup_\beta
            f^{-1}(B_\beta)$
        \item $f^{-1}\left(\bigcap_\beta B_\beta\right) = \bigcap_\beta
            f^{-1}(B_\beta)$
        \item $f^{-1}(Y \setminus D) = X \ f^{-1}(D)$
        \item $f(f^{-1}(D)) = f(X) \cap D$
    \end{itemize}
    \begin{proof}
        Doof.
    \end{proof}
\end{lem}

\begin{dfn}
    $(X,\tau)$, $(Y, \tau')$ topologische R"aume, $f: X\to Y$ eine Abbildung.
    $f$ hei"st \emph{Quotientenabbildung}, falls
    \begin{itemize}
        \item $f$ surjektiv
        \item f"ur $U \subset Y$ gilt: $U\ \text{offen} \Equiv f^{-1}(U)\
            \mathrm{offen}$
    \end{itemize}

    \begin{bem}
        \begin{itemize}
            \item Eine Quotientenabbildung ist stetig.
            \item Eine stetige surjektive Abbildung ist im Allgemeinen
                \emph{kein} Quotient. ($(X, \tau_1) \overset{id} \to
                (X,\tau_2)$ mit $\tau_2$ strikt feiner als $\tau_2$)
        \end{itemize}
    \end{bem}

    \begin{bsps}
        \begin{itemize}
            \item Ist $f:(X,\tau)\to (Y,\tau')$ stetig, surjektiv und offen
                oder abgeschlossen, dann ist $f$ ein Quotient
            \item Sei $X=[0,1]\cup [2,3] \subset \R$, $Y=[0,2]\subset \R$ mit
                der euklidischen Topologie. Dann ist $f(x)=x\chi_{[0,1]} +
                (x-1)\chi_{[2,3]}$ stetig, surjektiv und abgeschlossen, also
                ein Quotient
            \item Seien $X$, $Y$ topologische R"aume, dann ist $X\times Y
                \overset[p_X]{\to} X$ ein Quotient
        \end{itemize}
    \end{bsps}
\end{dfn}

\begin{dfn}
    $(X,\tau)$ topologischer Raum, $Y$ eine Menge und $f:X\to Y$ eine
    surjektive Abbildung. Dann existiert genau eine Topologie $\tau_f$ auf $Y$,
    so dass $f$ eine Quotientenabbildung ist. Sie ist durch $\tau_f =
    \mdef[U\subset Y]{f^{-1}(U)\in \tau}$ definiert.
    \begin{proof}[Existenz]
        Lemma 37 (c) und (d)
    \end{proof}
\end{dfn}

\begin{lem}
    $(X, \tau)$, $(Y, \tau')$ topologische R"aume, $f:X\to Y$ surjektiv und
    $g:Y\to Z$ eine Abbildung. Dann sind "aquivalent:
    \begin{enumerate}
        \item $g:(Y,\tau_f)\to (Z, \tau')$ ist stetig
        \item $g\circ f : (X, \tau) \to (Z,\tau')$ ist stetig
    \end{enumerate}

    \begin{proof}
        Sei $U\subset Z$. Dann gilt:
        \begin{align}
            (g\circ f)^{-1}(U)=f^{-1}(g^{-1}(U) \ \mathrm{und} \\ g^{-1}(U)\
            \mathrm{ist offen}\ \Equiv\ f^{-1}g^{-1} \ \mathrm{ist offen}
        \end{align}
    \end{proof}
\end{lem}

\begin{dfn}
    Sei $R$ eine "Aquivalenzrelation und $\tau$ eine Topologie auf $X$, $\pi$
    die Projektion $X\to X/R$. Die Topologie $\tau_\pi$ auf $X/R$ hei"st
    \emph{Quotiententopologie}. $\pi$ hei"st \emph{kanonische Projektion}.
\end{dfn}

\begin{stz}[Universelle Eigenschaft]
    Ist $f:X\to Y$ eine stetige Abbildung, $R$ "Aquivalenzrelation auf $X$.
    Falls $x\sim y \Rightarrow f(x) = f(y)$, so gibt es genau eine Abbildung
    $\overline{f}$, so dass
    \[
    \begin{diagram}
        \node{X} \arrow{e,t}{f} \arrow{s,l}{\pi} \node{Y} \\
        \node{X/R} \arrow{ne, r}{\overline{f}}
    \end{diagram}
    \]
    kommutiert.

    \begin{bem}
        \[
        \begin{diagram}
            \node{X} \arrow{e,A,t}{f \mathrm{ surj.}} \arrow{s,A} \node{Y} \\
            \node{X/R} \arrow{ne,..,r}{\overline{f}}
        \end{diagram}
        \]
        Falls $f(x) = f(y) \Equiv x\sim y$, so ist $\overline{f}$ bijektiv aber
        im Allgemeinen \emph{kein} Hom"oomorphismus.
    \end{bem}
\end{stz}

\begin{dfn}
    $A\subset X$ Teilraum, $R_A$ die zur Partition $X=A\cup \bigcup_{x\in
    X\setminus A} \mdef{x}$ assoziierte "Aquivalenzrelation. Dann schreiben wir
    \[
    X/A := X/R_A
    \]
    \begin{bsp}
        \begin{gather}
            X = [0,1]\subset \R \\
            S^1 = \mdef[z\in\C]{\left|z\right|=1}\subset \C \cong \R^2 \\
            A = \mdef{0,1}
        \end{gather}
        Dann ist $X/A \cong S^1$. Ist $\phi:X\to Y$ ein Quotient und bijektiv,
        so ist $\phi$ ein Hom"oomorphismus. Die Einschr"ankung $\exp : [0,1)
        \to S^1$ ist bijektiv und die Einschr"ankung eines Quotienten aber
        nicht offen (also kein Hom"oomorphismus, also kein Quotient).

        \subsubsection*{Frage:} Wie vertr"aglich ist die Qutienteneigenschaft
        mit anderen Konstruktionen?
    \end{bsp}
\end{dfn}

\begin{dfn}
    Sei  $f:X\to Y$ eine Abbildung von Mengen. Eine Teilmenge $A\subset X$ heißt
    \emph{saturiert bezüglich f}, falls $A=f^{-1}f(A)$ (die Inklusion gilt
    immer).

    (Äquivalent: $\exists B\subset Y: f^{-1}(B) = A$)
\end{dfn}

\begin{stz}
    Sei $f:X\to Y$ eine Abbildung von topologischen Räumen. Dann sind
    äquivalent:
    \begin{enumerate}
        \item $f$ ist Quotientenabbildung
        \item $f$ ist stetig, surjektiv und bildet offene saturierte Teilmengen
            $U\subset X$ auf offene Teilmengen ab
        \item $f$ ist stetig, surjektiv und bildet abgeschlossene saturierte
            Teilmengen $A\subset X$ auf abgeschlossene Teilmengen ab
    \end{enumerate}
    \begin{proof}
        \begin{description}
            \item[$1 \Rightarrow 2$] Sei $A\subset X$ offen und saturiert. Dann
                gilt $A = f^{-1}(f(A))$ also $f(A)$ offen
            \item[$2 \Rightarrow 2$] Sei $U\subset Y$ mit $f^{-1}(U)$ offen. Da
                $f^{-1}(U)$ saturiert ist, gilt $U \overset{\mathrm{surj.}}{=}
                f(f^{-1}(U))$ offen
            \item[$1 \Equiv 3$] offen $\Equiv$ abgeschlossen
        \end{description}
    \end{proof}
\end{stz}

\begin{stz}
    Sei $f:X\to Y$ eine Quotientenabbildung zwischen topologischen Räumen. Seien
    $A\subset X$, $f(A)\subset Y$ mit Teilraumtopologien versehen und sei $g =
    f|_A$.
    \begin{enumerate}
        \item Ist $A$ offen oder abgeschlossen in $X$ und saturiert bezüglich
            $f$, dann ist $g$ eine Quotientenabbildung
        \item Ist $A$ saturiert und ist $f$ offen oder abgeschlossen, so ist $g$
            eine Quotientenabbildung
    \end{enumerate}
    \begin{proof}
        \begin{enumerate}
            \item Sei $U\subset A$, sodass $U$ saturiert bezüglich $g$ ist. Wir
                haben $U\subset A$, also $f^{-1}(f(U))\subset f^{-1}(f(A)) = A$.
                Also gilt $U = g^{-1}(g(U)) = g^{-1}(f(U)) = A \cap
                f^{-1}(f(U)) = f^{-1}(f(U))$ also ist $U$ saturiert bezüglich
                $f$.
                
                Sei $A$ offen. Ist $U\subset A$ offen und saturiert bezüglich
                $g$, so ist $U$ auch offen $X$ und saturiert bezüglich $f$. Da
                $f$ Quotient ist, gilt $g(U) = f(U)$ ist offen in $Y$ also auch
                in $f(A)$.
            \item Sei $V\subset X$. Allgemein gilt $f(A\cap V)\subset f(A) \cap
                f(V)$ und $A$ saturiert $\Rightarrow$ Gleichheit.

                Sei $U\subset A$ offen bezüglich $A$, also $f(U) = f(A\cap V) =
                f(A) \cap f(v)$ $\Rightarrow$ g offen $\Rightarrow$ Quotient.

                Ist $f$ offen, so ist $f(V)$ offen, also $f(U) = f(A) \cap f(V)$
                offen in $f(A)$.
        \end{enumerate}
    \end{proof}
    \begin{bsp}[Produkt zweier Quotienten nicht immer Produkt]
        $\R$ euklidisch, $\Q\subset \R$ Teilraum, $A = \N$.

        Sei $p:\R\to\R/\N$ der Quotient und $\id : \Q\to\Q$ (auch
        ein Quotient).

        Dann ist $p\times\id:\R\times\Q\to\R/\N$ bezüglich der Produkttopologie
        kein Quotient. Hierzu definiert man eine offene saturierte Teilmenge
        $U\subset\R\times\Q$ so dass $(p\times\id)(U)$ nicht offen ist
        $\Rightarrow$ (ref. bla) $p\times\id$ ist kein Quotient.
        \begin{align}
            U &= \bigcup_{n\in\N} \\
            U_n &= (\R\times\Q)\cap \overline{U_n},\ \overline{U}
                    \subset\R\times\R \text{ offen} \\
            W_n^\pm &= \mdef[(x,y)\in\R^2]{y \gtrless x - n +
                    \frac{\sqrt 2}{n}} \\
            Y_n^\pm &= \mdef[(x,y)\in\R^2]{y \gtrless n - x +
                    \frac{\sqrt 2}{n}} \\
            \overline{U_n}' &= (W^+_n \cap Y^+_n) \cup (W^-_n \cap Y^-_n) \\
            \overline{U_n} &= \mdef[(x,y)]{n-\frac{1}{4}<x<n+\frac{1}{4}} \cap
                    \overline{U_n}'
        \end{align}
        $U$ ist offen, weil $\overline{U_n} \subset \R\times\R$ ist.

        $U$ ist saturiert:\\
        Sei $z\in\R\times\Q$ mit $(\underset{f}{\underbrace{p\times\id}})\in
        f(U)$. $\exists y\in U$ mit $f(z) = f(y)$.
        \begin{description}
            \item[$y=z$,] ok.
            \item[$y\neq z$,] dann gilt: $y = (n, y_2)$, $z = (m,y_2)$ für
                $n,m\in\N_+$, also $z\in U$ (nicht saturiert in $\R\times\R$,
                $\frac{\sqrt 2}{n}\not\in\Q$)
        \end{description}

        Bild von $U$ ist nicht offen:\\
        Sei $x=(1,0)\in\R/\N\times\Q$. Wir finden eine Folge
        $(x_n)\subset(\R/\N\times\Q)\setminus f(U)$ mit $x\in\lim_n x_i$. Wähle
        $x_n\in B_{\frac{1}{n}}\left( n, \frac{\sqrt 2}{n} \right)\cap (
        (\R\times\Q)\setminus U)$.
    \end{bsp}
\end{stz}

\begin{stz}
    Sind $f_1:X_1\to Y_1$ und $f_2:X_2\to Y_2$ offene Abbildungen, dann ist
    $f_1\times f_2$ auch offen. Insbesondere folgt aus $f_i$ offen, surjektiv
    und stetig auch, dass $f_1\times f_2$ ein Quotient ist.
    \begin{proof}
        Ähnlich Aufgabe 4.\!3.
    \end{proof}
\end{stz}
\begin{stz}
    $X\overset{p}{\to}Y\overset{q}{\to}Z$ Quotientenabbildungen, dann ist auch
    $q\circ p$ ein Quotient.
\end{stz}

\begin{bsps}[Quotienten]
    \begin{enumerate}
        \item
            \begin{align}
                \intertext{Sei:} \\
                D^n &= \mdef[x\in\R^n]{\norm{x}\leq 1} \\
                \overset{\cup}{S^{n-1}} &= \mdef[x\in\R^n]{\norm{x} = 1}
            \end{align}
            Dann sind $D^n/S^{n-1}$ und $S^n$ homöomorph.
        \item 
    \end{enumerate}<++>
\end{bsps}<++>
\end{document}
